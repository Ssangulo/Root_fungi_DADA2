% Options for packages loaded elsewhere
\PassOptionsToPackage{unicode}{hyperref}
\PassOptionsToPackage{hyphens}{url}
\PassOptionsToPackage{dvipsnames,svgnames,x11names}{xcolor}
%
\documentclass[
  letterpaper,
  DIV=11,
  numbers=noendperiod]{scrartcl}

\usepackage{amsmath,amssymb}
\usepackage{iftex}
\ifPDFTeX
  \usepackage[T1]{fontenc}
  \usepackage[utf8]{inputenc}
  \usepackage{textcomp} % provide euro and other symbols
\else % if luatex or xetex
  \usepackage{unicode-math}
  \defaultfontfeatures{Scale=MatchLowercase}
  \defaultfontfeatures[\rmfamily]{Ligatures=TeX,Scale=1}
\fi
\usepackage{lmodern}
\ifPDFTeX\else  
    % xetex/luatex font selection
\fi
% Use upquote if available, for straight quotes in verbatim environments
\IfFileExists{upquote.sty}{\usepackage{upquote}}{}
\IfFileExists{microtype.sty}{% use microtype if available
  \usepackage[]{microtype}
  \UseMicrotypeSet[protrusion]{basicmath} % disable protrusion for tt fonts
}{}
\makeatletter
\@ifundefined{KOMAClassName}{% if non-KOMA class
  \IfFileExists{parskip.sty}{%
    \usepackage{parskip}
  }{% else
    \setlength{\parindent}{0pt}
    \setlength{\parskip}{6pt plus 2pt minus 1pt}}
}{% if KOMA class
  \KOMAoptions{parskip=half}}
\makeatother
\usepackage{xcolor}
\setlength{\emergencystretch}{3em} % prevent overfull lines
\setcounter{secnumdepth}{-\maxdimen} % remove section numbering
% Make \paragraph and \subparagraph free-standing
\ifx\paragraph\undefined\else
  \let\oldparagraph\paragraph
  \renewcommand{\paragraph}[1]{\oldparagraph{#1}\mbox{}}
\fi
\ifx\subparagraph\undefined\else
  \let\oldsubparagraph\subparagraph
  \renewcommand{\subparagraph}[1]{\oldsubparagraph{#1}\mbox{}}
\fi

\usepackage{color}
\usepackage{fancyvrb}
\newcommand{\VerbBar}{|}
\newcommand{\VERB}{\Verb[commandchars=\\\{\}]}
\DefineVerbatimEnvironment{Highlighting}{Verbatim}{commandchars=\\\{\}}
% Add ',fontsize=\small' for more characters per line
\usepackage{framed}
\definecolor{shadecolor}{RGB}{241,243,245}
\newenvironment{Shaded}{\begin{snugshade}}{\end{snugshade}}
\newcommand{\AlertTok}[1]{\textcolor[rgb]{0.68,0.00,0.00}{#1}}
\newcommand{\AnnotationTok}[1]{\textcolor[rgb]{0.37,0.37,0.37}{#1}}
\newcommand{\AttributeTok}[1]{\textcolor[rgb]{0.40,0.45,0.13}{#1}}
\newcommand{\BaseNTok}[1]{\textcolor[rgb]{0.68,0.00,0.00}{#1}}
\newcommand{\BuiltInTok}[1]{\textcolor[rgb]{0.00,0.23,0.31}{#1}}
\newcommand{\CharTok}[1]{\textcolor[rgb]{0.13,0.47,0.30}{#1}}
\newcommand{\CommentTok}[1]{\textcolor[rgb]{0.37,0.37,0.37}{#1}}
\newcommand{\CommentVarTok}[1]{\textcolor[rgb]{0.37,0.37,0.37}{\textit{#1}}}
\newcommand{\ConstantTok}[1]{\textcolor[rgb]{0.56,0.35,0.01}{#1}}
\newcommand{\ControlFlowTok}[1]{\textcolor[rgb]{0.00,0.23,0.31}{#1}}
\newcommand{\DataTypeTok}[1]{\textcolor[rgb]{0.68,0.00,0.00}{#1}}
\newcommand{\DecValTok}[1]{\textcolor[rgb]{0.68,0.00,0.00}{#1}}
\newcommand{\DocumentationTok}[1]{\textcolor[rgb]{0.37,0.37,0.37}{\textit{#1}}}
\newcommand{\ErrorTok}[1]{\textcolor[rgb]{0.68,0.00,0.00}{#1}}
\newcommand{\ExtensionTok}[1]{\textcolor[rgb]{0.00,0.23,0.31}{#1}}
\newcommand{\FloatTok}[1]{\textcolor[rgb]{0.68,0.00,0.00}{#1}}
\newcommand{\FunctionTok}[1]{\textcolor[rgb]{0.28,0.35,0.67}{#1}}
\newcommand{\ImportTok}[1]{\textcolor[rgb]{0.00,0.46,0.62}{#1}}
\newcommand{\InformationTok}[1]{\textcolor[rgb]{0.37,0.37,0.37}{#1}}
\newcommand{\KeywordTok}[1]{\textcolor[rgb]{0.00,0.23,0.31}{#1}}
\newcommand{\NormalTok}[1]{\textcolor[rgb]{0.00,0.23,0.31}{#1}}
\newcommand{\OperatorTok}[1]{\textcolor[rgb]{0.37,0.37,0.37}{#1}}
\newcommand{\OtherTok}[1]{\textcolor[rgb]{0.00,0.23,0.31}{#1}}
\newcommand{\PreprocessorTok}[1]{\textcolor[rgb]{0.68,0.00,0.00}{#1}}
\newcommand{\RegionMarkerTok}[1]{\textcolor[rgb]{0.00,0.23,0.31}{#1}}
\newcommand{\SpecialCharTok}[1]{\textcolor[rgb]{0.37,0.37,0.37}{#1}}
\newcommand{\SpecialStringTok}[1]{\textcolor[rgb]{0.13,0.47,0.30}{#1}}
\newcommand{\StringTok}[1]{\textcolor[rgb]{0.13,0.47,0.30}{#1}}
\newcommand{\VariableTok}[1]{\textcolor[rgb]{0.07,0.07,0.07}{#1}}
\newcommand{\VerbatimStringTok}[1]{\textcolor[rgb]{0.13,0.47,0.30}{#1}}
\newcommand{\WarningTok}[1]{\textcolor[rgb]{0.37,0.37,0.37}{\textit{#1}}}

\providecommand{\tightlist}{%
  \setlength{\itemsep}{0pt}\setlength{\parskip}{0pt}}\usepackage{longtable,booktabs,array}
\usepackage{calc} % for calculating minipage widths
% Correct order of tables after \paragraph or \subparagraph
\usepackage{etoolbox}
\makeatletter
\patchcmd\longtable{\par}{\if@noskipsec\mbox{}\fi\par}{}{}
\makeatother
% Allow footnotes in longtable head/foot
\IfFileExists{footnotehyper.sty}{\usepackage{footnotehyper}}{\usepackage{footnote}}
\makesavenoteenv{longtable}
\usepackage{graphicx}
\makeatletter
\def\maxwidth{\ifdim\Gin@nat@width>\linewidth\linewidth\else\Gin@nat@width\fi}
\def\maxheight{\ifdim\Gin@nat@height>\textheight\textheight\else\Gin@nat@height\fi}
\makeatother
% Scale images if necessary, so that they will not overflow the page
% margins by default, and it is still possible to overwrite the defaults
% using explicit options in \includegraphics[width, height, ...]{}
\setkeys{Gin}{width=\maxwidth,height=\maxheight,keepaspectratio}
% Set default figure placement to htbp
\makeatletter
\def\fps@figure{htbp}
\makeatother

\KOMAoption{captions}{tableheading}
\makeatletter
\@ifpackageloaded{caption}{}{\usepackage{caption}}
\AtBeginDocument{%
\ifdefined\contentsname
  \renewcommand*\contentsname{Table of contents}
\else
  \newcommand\contentsname{Table of contents}
\fi
\ifdefined\listfigurename
  \renewcommand*\listfigurename{List of Figures}
\else
  \newcommand\listfigurename{List of Figures}
\fi
\ifdefined\listtablename
  \renewcommand*\listtablename{List of Tables}
\else
  \newcommand\listtablename{List of Tables}
\fi
\ifdefined\figurename
  \renewcommand*\figurename{Figure}
\else
  \newcommand\figurename{Figure}
\fi
\ifdefined\tablename
  \renewcommand*\tablename{Table}
\else
  \newcommand\tablename{Table}
\fi
}
\@ifpackageloaded{float}{}{\usepackage{float}}
\floatstyle{ruled}
\@ifundefined{c@chapter}{\newfloat{codelisting}{h}{lop}}{\newfloat{codelisting}{h}{lop}[chapter]}
\floatname{codelisting}{Listing}
\newcommand*\listoflistings{\listof{codelisting}{List of Listings}}
\makeatother
\makeatletter
\makeatother
\makeatletter
\@ifpackageloaded{caption}{}{\usepackage{caption}}
\@ifpackageloaded{subcaption}{}{\usepackage{subcaption}}
\makeatother
\ifLuaTeX
  \usepackage{selnolig}  % disable illegal ligatures
\fi
\usepackage{bookmark}

\IfFileExists{xurl.sty}{\usepackage{xurl}}{} % add URL line breaks if available
\urlstyle{same} % disable monospaced font for URLs
\hypersetup{
  pdftitle={Analysis\_pipeline},
  colorlinks=true,
  linkcolor={blue},
  filecolor={Maroon},
  citecolor={Blue},
  urlcolor={Blue},
  pdfcreator={LaTeX via pandoc}}

\title{Analysis\_pipeline}
\author{}
\date{}

\begin{document}
\maketitle

\subsection{Overview and
reproducibility}\label{overview-and-reproducibility}

This document compiles methodological details, supplementary analyses,
and reproducibility outputs supporting the main manuscript.
Computationally intensive steps (e.g.~phylogenetic tree inference, GLLVM
fitting, iNEXT3D diversity estimation) were executed on a remote
Ubuntu-based computing environment. Their outputs (figures, tables, and
selected model objects) are integrated here as static results.

All figures (PNG) and result tables (CSV) are stored within the project
repository using relative paths. Rendering this document therefore
requires cloning the repository and opening the .qmd file from the
repository root.

\textbf{Data availability.}

Repository: To be provided upon acceptance\\
Sequence data: ENA accession number PRJEB107725\\
Metadata: \texttt{Data\_S1\_metadata.csv,\ Data\_S3\_soil.csv}

\begin{Shaded}
\begin{Highlighting}[]
\CommentTok{\# Output folders used by table/figure chunks}
\NormalTok{out\_fig\_dir }\OtherTok{\textless{}{-}} \StringTok{"figures"}
\NormalTok{out\_tab\_dir }\OtherTok{\textless{}{-}} \StringTok{"tables"}
\NormalTok{out\_obj\_dir }\OtherTok{\textless{}{-}} \StringTok{"objects"}

\CommentTok{\# Main analysis phyloseq object (root{-}only; replicates collapsed; soil reads removed; decontam applied)}
\NormalTok{ps\_individual }\OtherTok{\textless{}{-}} \FunctionTok{readRDS}\NormalTok{(}\FunctionTok{file.path}\NormalTok{(out\_obj\_dir, }\StringTok{"ps\_individual.rds"}\NormalTok{))}


\CommentTok{\# Phyloseq object that includes the phylogeny and keeping the  2 root replicates separate}
\NormalTok{tree\_ps  }\OtherTok{\textless{}{-}} \FunctionTok{readRDS}\NormalTok{(}\FunctionTok{file.path}\NormalTok{(out\_obj\_dir, }\StringTok{"tree\_ps.rds"}\NormalTok{))}
\end{Highlighting}
\end{Shaded}

\subsection{Study sites and sampling}\label{study-sites-and-sampling}

This study was conducted across four humid páramo regions in the
Colombian Andes, spanning the Central and Eastern Cordilleras. Sampling
locations covered elevations from 2,715 to 3,828 m a.s.l., representing
the ecological transition from Andean forest to páramo (see main text,
Fig. 1). The full sample metadata table is provided in
Data\_S1\_metadata.csv.

Root sampling followed standardized sterile procedures. Fine roots of
Gaultheria myrsinoides were excavated using a spade, and 3--5 cm root
tips were clipped using a sterilized pruner. Tools were disinfected with
75\% ethanol between samples, and gloves were changed between individual
plants to minimize contamination. Root fragments were placed into sealed
polybags stored in ice-cooled containers in the field. Samples were
processed at Universidad ICESI, where roots were rinsed in sterile water
to remove debris, dried briefly under a laminar flow hood, and stored at
--20°C prior to shipment to Norway, and long-term storage at --80°C.
Soil samples (top 15 cm) were collected at each sampling location, and
five aliquots of rinse water were retained as laboratory and transport
controls.

\textbf{Table S1}: GPS coordinates, elevation, and vegetation type
classification of the 12 sampling locations across four páramo regions.
These sites span the forest--subpáramo--páramo gradient and were used
for root and soil sampling.

\begin{longtable}[]{@{}
  >{\raggedright\arraybackslash}p{(\columnwidth - 10\tabcolsep) * \real{0.1667}}
  >{\raggedright\arraybackslash}p{(\columnwidth - 10\tabcolsep) * \real{0.1667}}
  >{\raggedright\arraybackslash}p{(\columnwidth - 10\tabcolsep) * \real{0.1667}}
  >{\raggedright\arraybackslash}p{(\columnwidth - 10\tabcolsep) * \real{0.1667}}
  >{\raggedright\arraybackslash}p{(\columnwidth - 10\tabcolsep) * \real{0.1667}}
  >{\raggedright\arraybackslash}p{(\columnwidth - 10\tabcolsep) * \real{0.1667}}@{}}
\toprule\noalign{}
\endhead
\bottomrule\noalign{}
\endlastfoot
\textbf{Site ID} & \textbf{Site} & \textbf{Altitude (m)} &
\textbf{Latitude (N)} & \textbf{Longitude (W)} & \textbf{Ecosystem} \\
NV\_1 & Páramo La Nevera & 3828 & 03º30'58.1'' & 076º03'05.4'' &
Páramo \\
NV\_2 & Páramo La Nevera & 3550 & 03º31'37.7'' & 076º03'38.7'' &
Subpáramo \\
NV\_3 & Páramo La Nevera & 3333 & 03º32'07.0'' & 076º04'18.3'' & Andean
forest \\
NV\_4 & Páramo La Nevera & 3078 & 03º32'48.9'' & 076º04'31.3'' & Andean
forest \\
DOM\_1 & Páramo Las Domínguez & 3815 & 03º43'51.8'' & 076º06'38.1'' &
Páramo \\
DOM\_2 & Páramo Las Domínguez & 3534 & 03º44'07.2'' & 076º05'53.2'' &
Subpáramo \\
DOM\_3 & Páramo Las Domínguez & 3234 & 03º44'17.1'' & 076º05'32.9'' &
Andean forest \\
BEL\_1 & Páramo Belmira & 3254 & 06º38'43.8'' & 075º40'13.3'' &
Subpáramo \\
BEL\_2 & Páramo Belmira & 2950 & 06º38'22.7'' & 075º39'57.3'' & Andean
forest \\
BEL\_3 & Páramo Belmira & 2715 & 06°36'55.3'' & 075°39'51.1'' & Andean
forest \\
MA\_1 & Páramo Matarredonda & 3678 & 04º33'31.5'' & 074º01'43.8'' &
Páramo \\
MA\_2 & Páramo Matarredonda & 3381 & 04º33'10.3'' & 073º59'35.3'' &
Subpáramo \\
\end{longtable}

\subsection{Bioinformatic pipeline and read-filtering
summary}\label{bioinformatic-pipeline-and-read-filtering-summary}

This section provides supplementary details on sequencing read
processing and quality filtering. The full bioinformatic pipeline
(including DADA2 processing and replicate filtering) is available in the
project repository. Here we report summary tables of read retention and
replicate-filtering thresholds used for downstream analyses.

\textbf{Table S2}: Median read numbers per PCR replicate for
environmental samples (root and soil samples) and controls (blank
extractions, positive, PCR blank, and tag-jump controls). The read
counts are grouped at each stage of the DADA2 pipeline. Blank extraction
controls (not containing any biological sample) were performed alongside
real sample DNA extractions per each extraction batch. PCR blank
controls contained all PCR reagents but no DNA template. Tag-jump
controls are PCR reactions containing no primers and no DNA template,
used to detect tag-jumping (index hopping).~

\begin{longtable}[]{@{}
  >{\raggedright\arraybackslash}p{(\columnwidth - 10\tabcolsep) * \real{0.1667}}
  >{\raggedright\arraybackslash}p{(\columnwidth - 10\tabcolsep) * \real{0.1667}}
  >{\raggedright\arraybackslash}p{(\columnwidth - 10\tabcolsep) * \real{0.1667}}
  >{\raggedright\arraybackslash}p{(\columnwidth - 10\tabcolsep) * \real{0.1667}}
  >{\raggedright\arraybackslash}p{(\columnwidth - 10\tabcolsep) * \real{0.1667}}
  >{\raggedright\arraybackslash}p{(\columnwidth - 10\tabcolsep) * \real{0.1667}}@{}}
\toprule\noalign{}
\endhead
\bottomrule\noalign{}
\endlastfoot
& \textbf{Raw reads} & \textbf{Filtered} & \textbf{Merged - Non pooled}
& \textbf{Merged - Pooled} & \textbf{Merged - Pseudopooled} \\
\textbf{Root samples} & 77,047 & 71,783 & 71,084 & 70,343 & 71,242 \\
\textbf{Soil samples} & 72,158 & 66,555 & 65,311 & 65,498 & 65,385 \\
\textbf{Blank Extraction Controls} & 2,844 & 2,652 & 2,618 & 2,566 &
2,625 \\
\textbf{Positive Controls} & 95,238 & 88,174 & 88,062 & 87,996 &
87,972 \\
\textbf{PCR Blank Controls} & 291 & 261 & 250 & 257 & 256 \\
\textbf{Tag-jump Controls} & 41 & 36 & 27 & 30 & 32 \\
\end{longtable}

\begin{center}\rule{0.5\linewidth}{0.5pt}\end{center}

\begin{verbatim}
Warning: package 'knitr' was built under R version 4.4.3
\end{verbatim}

\begin{longtable}[]{@{}
  >{\raggedright\arraybackslash}p{(\columnwidth - 8\tabcolsep) * \real{0.2000}}
  >{\raggedleft\arraybackslash}p{(\columnwidth - 8\tabcolsep) * \real{0.2105}}
  >{\raggedleft\arraybackslash}p{(\columnwidth - 8\tabcolsep) * \real{0.1895}}
  >{\raggedleft\arraybackslash}p{(\columnwidth - 8\tabcolsep) * \real{0.2105}}
  >{\raggedleft\arraybackslash}p{(\columnwidth - 8\tabcolsep) * \real{0.1895}}@{}}
\caption{Table S3. Number of OTUs and samples after applying different
replicate control filtering thresholds. The table shows results for both
soil and root samples combined and only root samples.}\tabularnewline
\toprule\noalign{}
\begin{minipage}[b]{\linewidth}\raggedright
Filtering\_level
\end{minipage} & \begin{minipage}[b]{\linewidth}\raggedleft
OTU\_count\_soil+root
\end{minipage} & \begin{minipage}[b]{\linewidth}\raggedleft
Samples\_soil+root
\end{minipage} & \begin{minipage}[b]{\linewidth}\raggedleft
OTU\_count\_root\_only
\end{minipage} & \begin{minipage}[b]{\linewidth}\raggedleft
Samples\_root\_only
\end{minipage} \\
\midrule\noalign{}
\endfirsthead
\toprule\noalign{}
\begin{minipage}[b]{\linewidth}\raggedright
Filtering\_level
\end{minipage} & \begin{minipage}[b]{\linewidth}\raggedleft
OTU\_count\_soil+root
\end{minipage} & \begin{minipage}[b]{\linewidth}\raggedleft
Samples\_soil+root
\end{minipage} & \begin{minipage}[b]{\linewidth}\raggedleft
OTU\_count\_root\_only
\end{minipage} & \begin{minipage}[b]{\linewidth}\raggedleft
Samples\_root\_only
\end{minipage} \\
\midrule\noalign{}
\endhead
\bottomrule\noalign{}
\endlastfoot
No filtering & 18451 & 132 & 13225 & 120 \\
OTU present in 2/4 & 7174 & 132 & 4386 & 120 \\
OTU present in 3/4 & 4870 & 131 & 2966 & 120 \\
OTU present in 4/4 & 2875 & 113 & 1904 & 103 \\
\end{longtable}

\begin{center}\rule{0.5\linewidth}{0.5pt}\end{center}

\textbf{Table S4}: Summary of the percentage of reads and OTUS with
confident taxonomic assignment at each taxonomic rank, represented by
habitat. Percentages were calculated from the filtered root-only dataset
(soil reads removed).

\begin{longtable}[]{@{}
  >{\raggedright\arraybackslash}p{(\columnwidth - 6\tabcolsep) * \real{0.1471}}
  >{\raggedright\arraybackslash}p{(\columnwidth - 6\tabcolsep) * \real{0.1912}}
  >{\raggedright\arraybackslash}p{(\columnwidth - 6\tabcolsep) * \real{0.3382}}
  >{\raggedright\arraybackslash}p{(\columnwidth - 6\tabcolsep) * \real{0.3235}}@{}}
\toprule\noalign{}
\endhead
\bottomrule\noalign{}
\endlastfoot
\textbf{Rank} & \textbf{Habitat} & \textbf{\%\_reads\_assigned} &
\textbf{\%\_otus\_assigned} \\
Phylum & Forest & 88,1 & 88,1 \\
Phylum & Subpáramo & 91,6 & 88,6 \\
Phylum & Páramo & 92,9 & 89,3 \\
Class & Forest & 78,3 & 72,5 \\
Class & Subpáramo & 84,2 & 72,3 \\
Class & Páramo & 84,3 & 76,0 \\
Order & Forest & 68,3 & 59,2 \\
Order & Subpáramo & 74,2 & 57,1 \\
Order & Páramo & 78,2 & 60,9 \\
Genus & Forest & 32,8 & 34,2 \\
Genus & Subpáramo & 32,1 & 32,2 \\
Genus & Páramo & 31,4 & 32,7 \\
\end{longtable}

\subsection{Taxonomic composition of root and soil
communities}\label{taxonomic-composition-of-root-and-soil-communities}

\includegraphics{figures/FigS1_Venn.png}

\textbf{Figure S1}: Venn diagram of taxonomic assignments across
sampling sites, in the root only dataset, but previous to soil sequences
removal. Venn diagrams illustrate the overlap of OTU taxonomic
assignments across four study sites: La Nevera (NV), Las Domínguez
(DOM), Belmira (BEL), and Matarredonda (MA). Only OTUs with confirmed
assignments at each taxonomic level are included.

\begin{center}\rule{0.5\linewidth}{0.5pt}\end{center}

\includegraphics{figures/FigS2.png}

\textbf{Figure S2}: Taxonomic composition in root samples along the
elevational gradient. Relative read abundances are shown at two
taxonomic ranks:~ A) Class and B) Genus. Each bar represents an
individual root sample, grouped by habitat: Forest (2700-3300 m),
Subpáramo (3300-3500 m), and Páramo (3800 m). Study sites are indicated
on the y-axis. \emph{Code: see ``MicroViz code for taxonomic barplots''
below.}

\begin{center}\rule{0.5\linewidth}{0.5pt}\end{center}

\includegraphics{figures/FigS3.png}

\textbf{Figure S3}: Taxonomic composition in soil samples, collected at
each of the sampling locations. Relative read abundances are shown at
three taxonomic ranks:~ A) Phylum, B) Genus, and C) Order. Each bar
represents an individual root sample, grouped by habitat: Forest
(2700-3300 m), Subpáramo (3300-3500 m), and Páramo (3800 m). Study sites
are indicated on the y-axis. \emph{Code: see ``MicroViz code for
taxonomic barplots'' below.}

\begin{Shaded}
\begin{Highlighting}[]
\CommentTok{\# MicroViz taxonomic barplots}
\CommentTok{\# Code executed on remote server; plots saved as PNG and embedded above.}

\FunctionTok{library}\NormalTok{(microViz)}

\NormalTok{filter\_taxa\_by\_rank }\OtherTok{\textless{}{-}} \ControlFlowTok{function}\NormalTok{(physeq\_obj, }\AttributeTok{rank\_prefix =} \StringTok{"o\_\_"}\NormalTok{, }\AttributeTok{exclude\_term =} \StringTok{"Incertae\_sedis"}\NormalTok{) \{}
\NormalTok{  taxa\_assigned }\OtherTok{\textless{}{-}} \FunctionTok{grepl}\NormalTok{(rank\_prefix, }\FunctionTok{tax\_table}\NormalTok{(physeq\_obj)[, }\StringTok{"Order"}\NormalTok{])}
\NormalTok{  exclude\_incertae\_sedis }\OtherTok{\textless{}{-}} \SpecialCharTok{!}\FunctionTok{grepl}\NormalTok{(exclude\_term, }\FunctionTok{tax\_table}\NormalTok{(physeq\_obj)[, }\StringTok{"Order"}\NormalTok{])}
\NormalTok{  taxa\_filtered }\OtherTok{\textless{}{-}}\NormalTok{ taxa\_assigned }\SpecialCharTok{\&}\NormalTok{ exclude\_incertae\_sedis}
\NormalTok{  physeq\_filtered }\OtherTok{\textless{}{-}} \FunctionTok{prune\_taxa}\NormalTok{(taxa\_filtered, physeq\_obj)}
  \FunctionTok{return}\NormalTok{(physeq\_filtered)}
\NormalTok{\}}

\NormalTok{ps\_filtered }\OtherTok{\textless{}{-}} \FunctionTok{filter\_taxa\_by\_rank}\NormalTok{(alldat.N[[}\DecValTok{2}\NormalTok{]])}

\CommentTok{\# Clean taxonomic prefixes like o\_\_, g\_\_, etc.}
\FunctionTok{tax\_table}\NormalTok{(ps\_filtered) }\OtherTok{\textless{}{-}} \FunctionTok{apply}\NormalTok{(}
  \FunctionTok{tax\_table}\NormalTok{(ps\_filtered), }\DecValTok{2}\NormalTok{, }\ControlFlowTok{function}\NormalTok{(x) }\FunctionTok{gsub}\NormalTok{(}\StringTok{"\^{}[a{-}z]\_\_"}\NormalTok{, }\StringTok{""}\NormalTok{, x)}
\NormalTok{)}

\CommentTok{\# Recode habitat labels}
\FunctionTok{sample\_data}\NormalTok{(ps\_filtered)}\SpecialCharTok{$}\NormalTok{habitat }\OtherTok{\textless{}{-}}\NormalTok{ dplyr}\SpecialCharTok{::}\FunctionTok{recode}\NormalTok{(}
  \FunctionTok{as.character}\NormalTok{(}\FunctionTok{sample\_data}\NormalTok{(ps\_filtered)}\SpecialCharTok{$}\NormalTok{habitat),}
  \StringTok{"forest"}    \OtherTok{=} \StringTok{"Forest"}\NormalTok{,}
  \StringTok{"subparamo"} \OtherTok{=} \StringTok{"Subpáramo"}\NormalTok{,}
  \StringTok{"paramo"}    \OtherTok{=} \StringTok{"Páramo"}\NormalTok{,}
  \AttributeTok{.default =} \ConstantTok{NA\_character\_}
\NormalTok{)}


\CommentTok{\# Order the factor levels}
\FunctionTok{sample\_data}\NormalTok{(ps\_filtered)}\SpecialCharTok{$}\NormalTok{habitat }\OtherTok{\textless{}{-}} \FunctionTok{factor}\NormalTok{(}
  \FunctionTok{sample\_data}\NormalTok{(ps\_filtered)}\SpecialCharTok{$}\NormalTok{habitat,}
  \AttributeTok{levels =} \FunctionTok{c}\NormalTok{(}\StringTok{"Forest"}\NormalTok{,}\StringTok{"Subpáramo"}\NormalTok{,}\StringTok{"Páramo"}\NormalTok{),}
  \AttributeTok{ordered =} \ConstantTok{TRUE}
\NormalTok{)}

\NormalTok{ps\_filtered }\OtherTok{\textless{}{-}}\NormalTok{ ps\_filtered }\SpecialCharTok{\%\textgreater{}\%} \FunctionTok{ps\_arrange}\NormalTok{(site)}

\CommentTok{\#  within each site, order samples by Bray/OLO seriation at Order level {-}{-}{-}}
\NormalTok{site\_levels }\OtherTok{\textless{}{-}} \FunctionTok{sample\_data}\NormalTok{(ps\_filtered) }\SpecialCharTok{\%\textgreater{}\%} \FunctionTok{as.data.frame}\NormalTok{() }\SpecialCharTok{\%\textgreater{}\%} \FunctionTok{pull}\NormalTok{(site) }\SpecialCharTok{\%\textgreater{}\%} \FunctionTok{unique}\NormalTok{()}

\NormalTok{samp\_order }\OtherTok{\textless{}{-}} \FunctionTok{unlist}\NormalTok{(}\FunctionTok{lapply}\NormalTok{(site\_levels, }\ControlFlowTok{function}\NormalTok{(s) \{}
\NormalTok{  ps\_filtered }\SpecialCharTok{\%\textgreater{}\%}
    \FunctionTok{ps\_filter}\NormalTok{(site }\SpecialCharTok{==}\NormalTok{ s) }\SpecialCharTok{\%\textgreater{}\%}
    \FunctionTok{ps\_seriate}\NormalTok{(}\AttributeTok{rank =} \StringTok{"Order"}\NormalTok{) }\SpecialCharTok{\%\textgreater{}\%}   \CommentTok{\# same tax\_level you plot}
    \FunctionTok{sample\_names}\NormalTok{()                   }\CommentTok{\# returns samples in seriated order}
\NormalTok{\}))}

\CommentTok{\# Barplot faceted by habitat, clean legend labels}
\NormalTok{p }\OtherTok{\textless{}{-}}\NormalTok{ ps\_filtered }\SpecialCharTok{\%\textgreater{}\%}
  \FunctionTok{comp\_barplot}\NormalTok{(}
    \AttributeTok{tax\_level =} \StringTok{"Order"}\NormalTok{, }\AttributeTok{n\_taxa =} \DecValTok{10}\NormalTok{, }\AttributeTok{label =} \StringTok{"site"}\NormalTok{,}
    \AttributeTok{bar\_outline\_colour =} \StringTok{"grey5"}\NormalTok{, }\AttributeTok{facet\_by =} \StringTok{"habitat"}\NormalTok{, }\AttributeTok{sample\_order =}\NormalTok{ samp\_order,   }
    \AttributeTok{merge\_other =} \ConstantTok{FALSE}\NormalTok{, }\AttributeTok{other\_name =} \StringTok{"Other orders"}
\NormalTok{  ) }\SpecialCharTok{+}
  \FunctionTok{coord\_flip}\NormalTok{() }\SpecialCharTok{+}
  \FunctionTok{theme}\NormalTok{(}
    \AttributeTok{legend.text  =} \FunctionTok{element\_text}\NormalTok{(}\AttributeTok{size =} \DecValTok{6}\NormalTok{),}
    \AttributeTok{legend.title =} \FunctionTok{element\_text}\NormalTok{(}\AttributeTok{size =} \DecValTok{7}\NormalTok{),}
    \AttributeTok{legend.key.size =} \FunctionTok{unit}\NormalTok{(}\FloatTok{0.3}\NormalTok{, }\StringTok{"cm"}\NormalTok{),}
    \AttributeTok{legend.spacing.x =} \FunctionTok{unit}\NormalTok{(}\FloatTok{0.2}\NormalTok{, }\StringTok{"cm"}\NormalTok{),}
    \AttributeTok{plot.title =} \FunctionTok{element\_text}\NormalTok{(}\AttributeTok{hjust =} \FloatTok{0.5}\NormalTok{)}
\NormalTok{  )}




\CommentTok{\#Plot for phylum}
\NormalTok{phy }\OtherTok{\textless{}{-}}\NormalTok{ ps\_filtered }\SpecialCharTok{\%\textgreater{}\%}
  \FunctionTok{comp\_barplot}\NormalTok{(}
    \AttributeTok{tax\_level =} \StringTok{"Phylum"}\NormalTok{, }\AttributeTok{n\_taxa =} \DecValTok{5}\NormalTok{, }\AttributeTok{label =} \StringTok{"site"}\NormalTok{,}
    \AttributeTok{bar\_outline\_colour =} \StringTok{"grey5"}\NormalTok{, }\AttributeTok{facet\_by =} \StringTok{"habitat"}\NormalTok{, }\AttributeTok{sample\_order =}\NormalTok{ samp\_order,   }
    \AttributeTok{merge\_other =} \ConstantTok{FALSE}\NormalTok{, }\AttributeTok{other\_name =} \StringTok{"Other phyla"}
\NormalTok{  ) }\SpecialCharTok{+}
  \FunctionTok{coord\_flip}\NormalTok{() }\SpecialCharTok{+}
  \FunctionTok{theme}\NormalTok{(}
    \AttributeTok{legend.text  =} \FunctionTok{element\_text}\NormalTok{(}\AttributeTok{size =} \DecValTok{6}\NormalTok{),}
    \AttributeTok{legend.title =} \FunctionTok{element\_text}\NormalTok{(}\AttributeTok{size =} \DecValTok{7}\NormalTok{),}
    \AttributeTok{legend.key.size =} \FunctionTok{unit}\NormalTok{(}\FloatTok{0.3}\NormalTok{, }\StringTok{"cm"}\NormalTok{),}
    \AttributeTok{legend.spacing.x =} \FunctionTok{unit}\NormalTok{(}\FloatTok{0.2}\NormalTok{, }\StringTok{"cm"}\NormalTok{),}
    \AttributeTok{plot.title =} \FunctionTok{element\_text}\NormalTok{(}\AttributeTok{hjust =} \FloatTok{0.5}\NormalTok{)}
\NormalTok{  )}



\CommentTok{\#Plot for class}
\NormalTok{class }\OtherTok{\textless{}{-}}\NormalTok{ ps\_filtered }\SpecialCharTok{\%\textgreater{}\%}
  \FunctionTok{comp\_barplot}\NormalTok{(}
    \AttributeTok{tax\_level =} \StringTok{"Class"}\NormalTok{, }\AttributeTok{n\_taxa =} \DecValTok{10}\NormalTok{, }\AttributeTok{label =} \StringTok{"site"}\NormalTok{,}
    \AttributeTok{bar\_outline\_colour =} \StringTok{"grey5"}\NormalTok{, }\AttributeTok{facet\_by =} \StringTok{"habitat"}\NormalTok{, }\AttributeTok{sample\_order =}\NormalTok{ samp\_order,   }
    \AttributeTok{merge\_other =} \ConstantTok{FALSE}\NormalTok{, }\AttributeTok{other\_name =} \StringTok{"Other classes"}
\NormalTok{  ) }\SpecialCharTok{+}
  \FunctionTok{coord\_flip}\NormalTok{() }\SpecialCharTok{+}
  \FunctionTok{theme}\NormalTok{(}
    \AttributeTok{legend.text  =} \FunctionTok{element\_text}\NormalTok{(}\AttributeTok{size =} \DecValTok{6}\NormalTok{),}
    \AttributeTok{legend.title =} \FunctionTok{element\_text}\NormalTok{(}\AttributeTok{size =} \DecValTok{7}\NormalTok{),}
    \AttributeTok{legend.key.size =} \FunctionTok{unit}\NormalTok{(}\FloatTok{0.3}\NormalTok{, }\StringTok{"cm"}\NormalTok{),}
    \AttributeTok{legend.spacing.x =} \FunctionTok{unit}\NormalTok{(}\FloatTok{0.2}\NormalTok{, }\StringTok{"cm"}\NormalTok{),}
    \AttributeTok{plot.title =} \FunctionTok{element\_text}\NormalTok{(}\AttributeTok{hjust =} \FloatTok{0.5}\NormalTok{)}
\NormalTok{  )}



\CommentTok{\#Plot for genus}
\NormalTok{genus }\OtherTok{\textless{}{-}}\NormalTok{ ps\_filtered }\SpecialCharTok{\%\textgreater{}\%}
  \FunctionTok{comp\_barplot}\NormalTok{(}
    \AttributeTok{tax\_level =} \StringTok{"Genus"}\NormalTok{, }\AttributeTok{n\_taxa =} \DecValTok{20}\NormalTok{, }\AttributeTok{label =} \StringTok{"site"}\NormalTok{,}
    \AttributeTok{bar\_outline\_colour =} \StringTok{"grey5"}\NormalTok{, }\AttributeTok{facet\_by =} \StringTok{"habitat"}\NormalTok{, }\AttributeTok{sample\_order =}\NormalTok{ samp\_order,   }
    \AttributeTok{merge\_other =} \ConstantTok{FALSE}\NormalTok{, }\AttributeTok{other\_name =} \StringTok{"Other genera"}
\NormalTok{  ) }\SpecialCharTok{+}
  \FunctionTok{coord\_flip}\NormalTok{() }\SpecialCharTok{+}
  \FunctionTok{theme}\NormalTok{(}
    \AttributeTok{legend.text  =} \FunctionTok{element\_text}\NormalTok{(}\AttributeTok{size =} \DecValTok{6}\NormalTok{),}
    \AttributeTok{legend.title =} \FunctionTok{element\_text}\NormalTok{(}\AttributeTok{size =} \DecValTok{7}\NormalTok{),}
    \AttributeTok{legend.key.size =} \FunctionTok{unit}\NormalTok{(}\FloatTok{0.3}\NormalTok{, }\StringTok{"cm"}\NormalTok{),}
    \AttributeTok{legend.spacing.x =} \FunctionTok{unit}\NormalTok{(}\FloatTok{0.2}\NormalTok{, }\StringTok{"cm"}\NormalTok{),}
    \AttributeTok{plot.title =} \FunctionTok{element\_text}\NormalTok{(}\AttributeTok{hjust =} \FloatTok{0.5}\NormalTok{)}
\NormalTok{  )}
\end{Highlighting}
\end{Shaded}

\section{Community composition
analyses}\label{community-composition-analyses}

This section provides detailed statistical analyses supporting the
patterns of root-associated fungal community composition described in
the main text. Several of these analyses form the basis of results
reported in the main manuscript; here we document the full model
specifications, test statistics, and complementary ordinations to ensure
transparency and reproducibility.

We include distance-based redundancy analysis (dbRDA) based on robust
Aitchison distances to assess habitat-associated compositional structure
while conditioning on site. We further present PERMANOVA results from a
balanced subset of samples (NV--DOM) used to test habitat and site
effects under a controlled sampling design. In addition, we report a
centroid-based turnover model quantifying the magnitude of community
change across habitats using the full dataset.

\subsubsection{Multivariate ordination}\label{multivariate-ordination}

Distance-based redundancy analysis (dbRDA) was conducted on robust
Aitchison distances to test for habitat-associated compositional
structure while conditioning on site.

\includegraphics{figures/Fig_S4.png}

\textbf{Figure S4}: Distance-based redundancy analysis (dbRDA) of
root-associated fungal communities based on robust Aitchison distances.
Points are colored by habitat; site was included as a conditioning
variable. \emph{Code: see ``Code used to generate Figure S4 and Table
S5'' below.}

\begin{center}\rule{0.5\linewidth}{0.5pt}\end{center}

\begin{longtable}[]{@{}lrrrr@{}}
\caption{Table S5a. dbRDA permutation tests (by term; 999
permutations).}\tabularnewline
\toprule\noalign{}
Component & Df & Variance & F & Pr \\
\midrule\noalign{}
\endfirsthead
\toprule\noalign{}
Component & Df & Variance & F & Pr \\
\midrule\noalign{}
\endhead
\bottomrule\noalign{}
\endlastfoot
habitat & 2 & 22.718 & 1.408 & 0.001 \\
Residual & 54 & 435.501 & NA & NA \\
\end{longtable}

\begin{longtable}[]{@{}lrrrr@{}}
\caption{Table S5b. dbRDA permutation tests (by constrained axis; 999
permutations).}\tabularnewline
\toprule\noalign{}
Axis & Df & Variance & F & Pr \\
\midrule\noalign{}
\endfirsthead
\toprule\noalign{}
Axis & Df & Variance & F & Pr \\
\midrule\noalign{}
\endhead
\bottomrule\noalign{}
\endlastfoot
CAP1 & 1 & 14.579 & 1.808 & 0.001 \\
CAP2 & 1 & 8.139 & 1.009 & 0.470 \\
Residual & 54 & 435.501 & NA & NA \\
\end{longtable}

\begin{Shaded}
\begin{Highlighting}[]
\CommentTok{\#Running dbRDA with robust aitchison}
\FunctionTok{library}\NormalTok{(phyloseq)}
\FunctionTok{library}\NormalTok{(vegan)}
\FunctionTok{library}\NormalTok{(knitr)}


\NormalTok{ps }\OtherTok{\textless{}{-}}\NormalTok{ ps\_individual }\CommentTok{\#phyloseq object}

\NormalTok{X }\OtherTok{\textless{}{-}} \FunctionTok{as}\NormalTok{(}\FunctionTok{otu\_table}\NormalTok{(ps), }\StringTok{"matrix"}\NormalTok{)}
\ControlFlowTok{if}\NormalTok{ (}\FunctionTok{taxa\_are\_rows}\NormalTok{(ps)) X }\OtherTok{\textless{}{-}} \FunctionTok{t}\NormalTok{(X)}

\NormalTok{sam }\OtherTok{\textless{}{-}} \FunctionTok{data.frame}\NormalTok{(}\FunctionTok{sample\_data}\NormalTok{(ps))}
\NormalTok{sam}\SpecialCharTok{$}\NormalTok{habitat }\OtherTok{\textless{}{-}} \FunctionTok{factor}\NormalTok{(sam}\SpecialCharTok{$}\NormalTok{habitat)}
\NormalTok{sam}\SpecialCharTok{$}\NormalTok{site    }\OtherTok{\textless{}{-}} \FunctionTok{factor}\NormalTok{(sam}\SpecialCharTok{$}\NormalTok{site)}


\CommentTok{\# dbRDA model: habitat constrained, conditioning on site}
\NormalTok{d\_ait }\OtherTok{\textless{}{-}} \FunctionTok{vegdist}\NormalTok{(}\FunctionTok{otu\_table}\NormalTok{(ps), }\AttributeTok{binary=}\ConstantTok{FALSE}\NormalTok{, }\AttributeTok{method=}\StringTok{"robust.aitchison"}\NormalTok{)}
\NormalTok{mod\_ait }\OtherTok{\textless{}{-}} \FunctionTok{capscale}\NormalTok{(d\_ait }\SpecialCharTok{\textasciitilde{}}\NormalTok{ habitat }\SpecialCharTok{+} \FunctionTok{Condition}\NormalTok{(site), }\AttributeTok{data =}\NormalTok{ sam)}

\CommentTok{\# Permutation tests (999 permutations)}
\FunctionTok{set.seed}\NormalTok{(}\DecValTok{1}\NormalTok{)}
\NormalTok{a\_terms   }\OtherTok{\textless{}{-}} \FunctionTok{anova.cca}\NormalTok{(mod\_ait, }\AttributeTok{by =} \StringTok{"terms"}\NormalTok{, }\AttributeTok{permutations =} \DecValTok{999}\NormalTok{)}
\NormalTok{a\_axis    }\OtherTok{\textless{}{-}} \FunctionTok{anova.cca}\NormalTok{(mod\_ait, }\AttributeTok{by =} \StringTok{"axis"}\NormalTok{, }\AttributeTok{permutations =} \DecValTok{999}\NormalTok{)}


\NormalTok{a\_terms}
\NormalTok{a\_axis}

\NormalTok{tab\_terms }\OtherTok{\textless{}{-}} \FunctionTok{data.frame}\NormalTok{(}
  \AttributeTok{Term =} \FunctionTok{rownames}\NormalTok{(a\_terms),}
  \AttributeTok{Df =}\NormalTok{ a\_terms}\SpecialCharTok{$}\NormalTok{Df,}
  \AttributeTok{Variance =}\NormalTok{ a\_terms}\SpecialCharTok{$}\NormalTok{Variance,}
  \AttributeTok{F =}\NormalTok{ a\_terms}\SpecialCharTok{$}\NormalTok{F,}
  \AttributeTok{Pr =}\NormalTok{ a\_terms}\SpecialCharTok{$}\StringTok{\textasciigrave{}}\AttributeTok{Pr(\textgreater{}F)}\StringTok{\textasciigrave{}}\NormalTok{,}
  \AttributeTok{row.names =} \ConstantTok{NULL}
\NormalTok{)}

\NormalTok{tab\_axis }\OtherTok{\textless{}{-}} \FunctionTok{data.frame}\NormalTok{(}
  \AttributeTok{Axis =} \FunctionTok{rownames}\NormalTok{(a\_axis),}
  \AttributeTok{Df =}\NormalTok{ a\_axis}\SpecialCharTok{$}\NormalTok{Df,}
  \AttributeTok{Variance =}\NormalTok{ a\_axis}\SpecialCharTok{$}\NormalTok{Variance,}
  \AttributeTok{F =}\NormalTok{ a\_axis}\SpecialCharTok{$}\NormalTok{F,}
  \AttributeTok{Pr =}\NormalTok{ a\_axis}\SpecialCharTok{$}\StringTok{\textasciigrave{}}\AttributeTok{Pr(\textgreater{}F)}\StringTok{\textasciigrave{}}\NormalTok{,}
  \AttributeTok{row.names =} \ConstantTok{NULL}
\NormalTok{)}

\FunctionTok{write.csv}\NormalTok{(tab\_terms, }\StringTok{"Table\_S4\_dbRDA\_terms.csv"}\NormalTok{, }\AttributeTok{row.names =} \ConstantTok{FALSE}\NormalTok{)}
\FunctionTok{write.csv}\NormalTok{(tab\_axis,  }\StringTok{"Table\_S4\_dbRDA\_axes.csv"}\NormalTok{,  }\AttributeTok{row.names =} \ConstantTok{FALSE}\NormalTok{)}


\CommentTok{\# Plotting Figure S4 }
\NormalTok{eig }\OtherTok{\textless{}{-}}\NormalTok{ mod\_ait}\SpecialCharTok{$}\NormalTok{CCA}\SpecialCharTok{$}\NormalTok{eig}
\NormalTok{cap1\_pct }\OtherTok{\textless{}{-}} \FunctionTok{round}\NormalTok{(}\DecValTok{100} \SpecialCharTok{*}\NormalTok{ eig[}\DecValTok{1}\NormalTok{] }\SpecialCharTok{/} \FunctionTok{sum}\NormalTok{(eig), }\DecValTok{1}\NormalTok{)}
\NormalTok{cap2\_pct }\OtherTok{\textless{}{-}} \FunctionTok{round}\NormalTok{(}\DecValTok{100} \SpecialCharTok{*}\NormalTok{ eig[}\DecValTok{2}\NormalTok{] }\SpecialCharTok{/} \FunctionTok{sum}\NormalTok{(eig), }\DecValTok{1}\NormalTok{)}

\NormalTok{cols }\OtherTok{\textless{}{-}} \FunctionTok{c}\NormalTok{(}\AttributeTok{forest=}\StringTok{"steelblue"}\NormalTok{, }\AttributeTok{subparamo=}\StringTok{"seagreen3"}\NormalTok{, }\AttributeTok{paramo=}\StringTok{"firebrick"}\NormalTok{)}

\FunctionTok{png}\NormalTok{(}\StringTok{"FigS4\_dbRDA.png"}\NormalTok{, }\AttributeTok{width=}\FloatTok{7.2}\NormalTok{, }\AttributeTok{height=}\FloatTok{6.2}\NormalTok{, }\AttributeTok{units=}\StringTok{"in"}\NormalTok{, }\AttributeTok{res=}\DecValTok{600}\NormalTok{)}
\FunctionTok{par}\NormalTok{(}\AttributeTok{mar=}\FunctionTok{c}\NormalTok{(}\FloatTok{4.5}\NormalTok{,}\FloatTok{4.5}\NormalTok{,}\DecValTok{1}\NormalTok{,}\DecValTok{1}\NormalTok{))}
\FunctionTok{plot}\NormalTok{(mod\_ait, }\AttributeTok{display=}\StringTok{"sites"}\NormalTok{, }\AttributeTok{type=}\StringTok{"n"}\NormalTok{,}
     \AttributeTok{xlab=}\FunctionTok{paste0}\NormalTok{(}\StringTok{"dbRDA1 ("}\NormalTok{,cap1\_pct,}\StringTok{"\%)"}\NormalTok{),}
     \AttributeTok{ylab=}\FunctionTok{paste0}\NormalTok{(}\StringTok{"dbRDA2 ("}\NormalTok{,cap2\_pct,}\StringTok{"\%)"}\NormalTok{))}
\NormalTok{pts }\OtherTok{\textless{}{-}} \FunctionTok{scores}\NormalTok{(mod\_ait, }\AttributeTok{display=}\StringTok{"sites"}\NormalTok{)}
\FunctionTok{points}\NormalTok{(pts, }\AttributeTok{pch=}\DecValTok{16}\NormalTok{, }\AttributeTok{cex=}\FloatTok{0.7}\NormalTok{, }\AttributeTok{col=}\NormalTok{cols[}\FunctionTok{as.character}\NormalTok{(sam}\SpecialCharTok{$}\NormalTok{habitat)])}
\FunctionTok{legend}\NormalTok{(}\StringTok{"topright"}\NormalTok{, }\AttributeTok{bty=}\StringTok{"n"}\NormalTok{, }\AttributeTok{legend=}\FunctionTok{levels}\NormalTok{(sam}\SpecialCharTok{$}\NormalTok{habitat),}
       \AttributeTok{col=}\NormalTok{cols[}\FunctionTok{levels}\NormalTok{(sam}\SpecialCharTok{$}\NormalTok{habitat)], }\AttributeTok{pch=}\DecValTok{16}\NormalTok{, }\AttributeTok{pt.cex=}\FloatTok{0.9}\NormalTok{)}
\FunctionTok{dev.off}\NormalTok{()}
\end{Highlighting}
\end{Shaded}

\subsubsection{PERMANOVA and dispersion
analyses}\label{permanova-and-dispersion-analyses}

Beta-dispersion tests indicated higher within-habitat variability in
forests compared with subpáramo and páramo communities (p \textless{}
0.001), whereas dispersion did not differ among sites (p = 0.526).
Because the PERMANOVA was conducted on a balanced subset of samples, the
significant effects of habitat, site, and their interaction are
interpreted as reflecting genuine compositional differences rather than
artifacts of unequal dispersion. \emph{Code: see ``Code used to generate
Tables S6a-S6c'' below.}

\begin{longtable}[]{@{}lrrrrr@{}}
\caption{Table S6a. PERMANOVA (adonis2) results by term.}\tabularnewline
\toprule\noalign{}
Source & df & SumOfSqs & R2 & Pseudo.F & Pr..F. \\
\midrule\noalign{}
\endfirsthead
\toprule\noalign{}
Source & df & SumOfSqs & R2 & Pseudo.F & Pr..F. \\
\midrule\noalign{}
\endhead
\bottomrule\noalign{}
\endlastfoot
site & 1 & 742.078 & 0.056 & 1.786 & 0.001 \\
habitat & 2 & 1351.164 & 0.102 & 1.626 & 0.001 \\
site:habitat & 2 & 1241.269 & 0.093 & 1.493 & 0.001 \\
Residual & 24 & 9974.546 & 0.749 & NA & NA \\
Total & 29 & 13309.057 & 1.000 & NA & NA \\
\end{longtable}

\begin{longtable}[]{@{}lr@{}}
\caption{Table S6b. Habitat beta-dispersion (distance to centroid).
Permutation test p = 0.001.}\tabularnewline
\toprule\noalign{}
habitat & avg\_distance\_to\_centroid \\
\midrule\noalign{}
\endfirsthead
\toprule\noalign{}
habitat & avg\_distance\_to\_centroid \\
\midrule\noalign{}
\endhead
\bottomrule\noalign{}
\endlastfoot
forest & 23.631 \\
paramo & 17.147 \\
subparamo & 17.794 \\
\end{longtable}

\begin{longtable}[]{@{}lrrr@{}}
\caption{Table S6c. Site beta-dispersion permutation
test.}\tabularnewline
\toprule\noalign{}
test & permutations & F & p\_value \\
\midrule\noalign{}
\endfirsthead
\toprule\noalign{}
test & permutations & F & p\_value \\
\midrule\noalign{}
\endhead
\bottomrule\noalign{}
\endlastfoot
betadisper\_site & 999 & 0.314 & 0.571 \\
\end{longtable}

\begin{Shaded}
\begin{Highlighting}[]
\NormalTok{ps }\OtherTok{\textless{}{-}}\NormalTok{ individual\_ps }\CommentTok{\#phyloseq obj}

\CommentTok{\# Balanced subset: NV + DOM, remove NV\_4}
\NormalTok{balanced\_ps }\OtherTok{\textless{}{-}} \FunctionTok{subset\_samples}\NormalTok{(ps, site }\SpecialCharTok{\%in\%} \FunctionTok{c}\NormalTok{(}\StringTok{"DOM"}\NormalTok{, }\StringTok{"NV"}\NormalTok{))}
\NormalTok{balanced\_ps }\OtherTok{\textless{}{-}} \FunctionTok{subset\_samples}\NormalTok{(balanced\_ps, site\_elevation }\SpecialCharTok{!=} \StringTok{"NV\_4"}\NormalTok{)}
\NormalTok{balanced\_ps }\OtherTok{\textless{}{-}} \FunctionTok{prune\_taxa}\NormalTok{(}\FunctionTok{taxa\_sums}\NormalTok{(balanced\_ps) }\SpecialCharTok{\textgreater{}} \DecValTok{0}\NormalTok{, balanced\_ps)}

\CommentTok{\# Ensure samples are rows}
\NormalTok{X }\OtherTok{\textless{}{-}} \FunctionTok{as}\NormalTok{(}\FunctionTok{otu\_table}\NormalTok{(balanced\_ps), }\StringTok{"matrix"}\NormalTok{)}
\ControlFlowTok{if}\NormalTok{ (}\FunctionTok{taxa\_are\_rows}\NormalTok{(balanced\_ps)) X }\OtherTok{\textless{}{-}} \FunctionTok{t}\NormalTok{(X)}

\NormalTok{sampledf }\OtherTok{\textless{}{-}} \FunctionTok{data.frame}\NormalTok{(}\FunctionTok{sample\_data}\NormalTok{(balanced\_ps))}
\NormalTok{sampledf}\SpecialCharTok{$}\NormalTok{site    }\OtherTok{\textless{}{-}} \FunctionTok{factor}\NormalTok{(sampledf}\SpecialCharTok{$}\NormalTok{site)}
\NormalTok{sampledf}\SpecialCharTok{$}\NormalTok{habitat }\OtherTok{\textless{}{-}} \FunctionTok{factor}\NormalTok{(sampledf}\SpecialCharTok{$}\NormalTok{habitat)}

\CommentTok{\# Robust Aitchison distance}
\NormalTok{dists }\OtherTok{\textless{}{-}} \FunctionTok{vegdist}\NormalTok{(X, }\AttributeTok{method =} \StringTok{"robust.aitchison"}\NormalTok{)}

\CommentTok{\# PERMANOVA (by terms)}
\FunctionTok{set.seed}\NormalTok{(}\DecValTok{1}\NormalTok{)}
\NormalTok{perma }\OtherTok{\textless{}{-}} \FunctionTok{adonis2}\NormalTok{(dists }\SpecialCharTok{\textasciitilde{}}\NormalTok{ site }\SpecialCharTok{*}\NormalTok{ habitat, }\AttributeTok{by =} \StringTok{"terms"}\NormalTok{, }\AttributeTok{data =}\NormalTok{ sampledf, }\AttributeTok{permutations =} \DecValTok{999}\NormalTok{)}

\CommentTok{\# Convert output to a clean CSV table}
\NormalTok{tab\_perma }\OtherTok{\textless{}{-}} \FunctionTok{data.frame}\NormalTok{(}
  \AttributeTok{Source   =} \FunctionTok{rownames}\NormalTok{(perma),}
  \AttributeTok{df       =}\NormalTok{ perma}\SpecialCharTok{$}\NormalTok{Df,}
  \AttributeTok{SumOfSqs =}\NormalTok{ perma}\SpecialCharTok{$}\NormalTok{SumOfSqs,}
  \AttributeTok{R2       =}\NormalTok{ perma}\SpecialCharTok{$}\NormalTok{R2,}
  \StringTok{\textasciigrave{}}\AttributeTok{Pseudo{-}F}\StringTok{\textasciigrave{}} \OtherTok{=}\NormalTok{ perma}\SpecialCharTok{$}\NormalTok{F,}
  \StringTok{\textasciigrave{}}\AttributeTok{Pr(\textgreater{}F)}\StringTok{\textasciigrave{}} \OtherTok{=}\NormalTok{ perma}\SpecialCharTok{$}\StringTok{\textasciigrave{}}\AttributeTok{Pr(\textgreater{}F)}\StringTok{\textasciigrave{}}\NormalTok{,}
  \AttributeTok{row.names =} \ConstantTok{NULL}
\NormalTok{)}

\CommentTok{\# Beta{-}dispersion (habitat \& site)}
\NormalTok{bd\_hab  }\OtherTok{\textless{}{-}} \FunctionTok{betadisper}\NormalTok{(dists, sampledf}\SpecialCharTok{$}\NormalTok{habitat)}
\NormalTok{bd\_site }\OtherTok{\textless{}{-}} \FunctionTok{betadisper}\NormalTok{(dists, sampledf}\SpecialCharTok{$}\NormalTok{site)}

\FunctionTok{set.seed}\NormalTok{(}\DecValTok{1}\NormalTok{)}
\NormalTok{pt\_hab  }\OtherTok{\textless{}{-}} \FunctionTok{permutest}\NormalTok{(bd\_hab,  }\AttributeTok{permutations =} \DecValTok{999}\NormalTok{)}
\NormalTok{pt\_site }\OtherTok{\textless{}{-}} \FunctionTok{permutest}\NormalTok{(bd\_site, }\AttributeTok{permutations =} \DecValTok{999}\NormalTok{)}

\CommentTok{\# Group dispersions }
\NormalTok{disp\_hab }\OtherTok{\textless{}{-}} \FunctionTok{data.frame}\NormalTok{(}
  \AttributeTok{habitat =} \FunctionTok{names}\NormalTok{(bd\_hab}\SpecialCharTok{$}\NormalTok{group.distances),}
  \AttributeTok{avg\_distance\_to\_centroid =} \FunctionTok{as.numeric}\NormalTok{(bd\_hab}\SpecialCharTok{$}\NormalTok{group.distances),}
  \AttributeTok{row.names =} \ConstantTok{NULL}
\NormalTok{)}

\NormalTok{disp\_site }\OtherTok{\textless{}{-}} \FunctionTok{data.frame}\NormalTok{(}
  \AttributeTok{site =} \FunctionTok{names}\NormalTok{(bd\_site}\SpecialCharTok{$}\NormalTok{group.distances),}
  \AttributeTok{avg\_distance\_to\_centroid =} \FunctionTok{as.numeric}\NormalTok{(bd\_site}\SpecialCharTok{$}\NormalTok{group.distances),}
  \AttributeTok{row.names =} \ConstantTok{NULL}
\NormalTok{)}


\NormalTok{test\_hab }\OtherTok{\textless{}{-}} \FunctionTok{data.frame}\NormalTok{(}
  \AttributeTok{test =} \StringTok{"betadisper\_habitat"}\NormalTok{,}
  \AttributeTok{permutations =} \DecValTok{999}\NormalTok{,}
  \AttributeTok{F =} \FunctionTok{unname}\NormalTok{(pt\_hab}\SpecialCharTok{$}\NormalTok{tab[}\DecValTok{1}\NormalTok{, }\StringTok{"F"}\NormalTok{]),}
  \AttributeTok{p\_value =} \FunctionTok{unname}\NormalTok{(pt\_hab}\SpecialCharTok{$}\NormalTok{tab[}\DecValTok{1}\NormalTok{, }\StringTok{"Pr(\textgreater{}F)"}\NormalTok{])}
\NormalTok{)}

\NormalTok{test\_site }\OtherTok{\textless{}{-}} \FunctionTok{data.frame}\NormalTok{(}
  \AttributeTok{test =} \StringTok{"betadisper\_site"}\NormalTok{,}
  \AttributeTok{permutations =} \DecValTok{999}\NormalTok{,}
  \AttributeTok{F =} \FunctionTok{unname}\NormalTok{(pt\_site}\SpecialCharTok{$}\NormalTok{tab[}\DecValTok{1}\NormalTok{, }\StringTok{"F"}\NormalTok{]),}
  \AttributeTok{p\_value =} \FunctionTok{unname}\NormalTok{(pt\_site}\SpecialCharTok{$}\NormalTok{tab[}\DecValTok{1}\NormalTok{, }\StringTok{"Pr(\textgreater{}F)"}\NormalTok{])}
\NormalTok{)}

\CommentTok{\# Output paths}
\NormalTok{out\_dir }\OtherTok{\textless{}{-}} \StringTok{"/data/lastexpansion/danieang/objects"}
\FunctionTok{dir.create}\NormalTok{(out\_dir, }\AttributeTok{showWarnings =} \ConstantTok{FALSE}\NormalTok{, }\AttributeTok{recursive =} \ConstantTok{TRUE}\NormalTok{)}

\FunctionTok{write.csv}\NormalTok{(tab\_perma,  }\FunctionTok{file.path}\NormalTok{(out\_dir, }\StringTok{"Table\_S6\_PERMANOVA.csv"}\NormalTok{), }\AttributeTok{row.names =} \ConstantTok{FALSE}\NormalTok{)}
\FunctionTok{write.csv}\NormalTok{(disp\_hab,   }\FunctionTok{file.path}\NormalTok{(out\_dir, }\StringTok{"Table\_S6\_dispersion\_habitat.csv"}\NormalTok{), }\AttributeTok{row.names =} \ConstantTok{FALSE}\NormalTok{)}
\FunctionTok{write.csv}\NormalTok{(disp\_site,  }\FunctionTok{file.path}\NormalTok{(out\_dir, }\StringTok{"Table\_S6\_dispersion\_site.csv"}\NormalTok{), }\AttributeTok{row.names =} \ConstantTok{FALSE}\NormalTok{)}
\FunctionTok{write.csv}\NormalTok{(test\_hab,   }\FunctionTok{file.path}\NormalTok{(out\_dir, }\StringTok{"Table\_S6\_betadisper\_test\_habitat.csv"}\NormalTok{), }\AttributeTok{row.names =} \ConstantTok{FALSE}\NormalTok{)}
\FunctionTok{write.csv}\NormalTok{(test\_site,  }\FunctionTok{file.path}\NormalTok{(out\_dir, }\StringTok{"Table\_S6\_betadisper\_test\_site.csv"}\NormalTok{), }\AttributeTok{row.names =} \ConstantTok{FALSE}\NormalTok{)}
\end{Highlighting}
\end{Shaded}

\subsection{Centroid-based turnover
model}\label{centroid-based-turnover-model}

To quantify the magnitude of community turnover across habitats while
accounting for spatial structure, we calculated the Aitchison distance
of each sample to its site-specific centroid and modelled turnover as a
function of habitat, site, and their interaction. Habitat had a strong
effect on turnover magnitude, whereas the habitat × site interaction was
not significant, indicating consistent habitat-associated turnover
across geographically isolated páramo regions (Table S7).

\textbf{Table S7}: Linear model results for centroid-based turnover.

\begin{longtable}[]{@{}lrrrrr@{}}
\toprule\noalign{}
Effect & df & Sum.of.Squares & Mean.Square & F.value & p.value \\
\midrule\noalign{}
\endhead
\bottomrule\noalign{}
\endlastfoot
habitat & 2 & 376.745 & 188.373 & 14.719 & 0.000 \\
site & 3 & 138.096 & 46.032 & 3.597 & 0.020 \\
habitat:site & 4 & 65.496 & 16.374 & 1.279 & 0.291 \\
Residuals & 50 & 639.914 & 12.798 & NA & NA \\
\end{longtable}

\begin{Shaded}
\begin{Highlighting}[]
\FunctionTok{library}\NormalTok{(phyloseq)}
\FunctionTok{library}\NormalTok{(vegan)}


\NormalTok{ps }\OtherTok{\textless{}{-}}\NormalTok{ individual\_ps }\CommentTok{\#phyloseq obj}


\CommentTok{\# OTU table {-}\textgreater{} samples x taxa}
\NormalTok{X }\OtherTok{\textless{}{-}} \FunctionTok{as}\NormalTok{(}\FunctionTok{otu\_table}\NormalTok{(ps), }\StringTok{"matrix"}\NormalTok{)}
\ControlFlowTok{if}\NormalTok{ (}\FunctionTok{taxa\_are\_rows}\NormalTok{(ps)) X }\OtherTok{\textless{}{-}} \FunctionTok{t}\NormalTok{(X)}
\NormalTok{X }\OtherTok{\textless{}{-}} \FunctionTok{as.matrix}\NormalTok{(X)}

\CommentTok{\# Robust CLR}
\NormalTok{rfy }\OtherTok{\textless{}{-}} \FunctionTok{decostand}\NormalTok{(X, }\StringTok{"rclr"}\NormalTok{, }\AttributeTok{MARGIN =} \DecValTok{1}\NormalTok{)}

\CommentTok{\# Metadata}
\NormalTok{meta }\OtherTok{\textless{}{-}} \FunctionTok{data.frame}\NormalTok{(}\FunctionTok{sample\_data}\NormalTok{(ps), }\AttributeTok{check.names =} \ConstantTok{FALSE}\NormalTok{, }\AttributeTok{stringsAsFactors =} \ConstantTok{FALSE}\NormalTok{)}
\FunctionTok{stopifnot}\NormalTok{(}\FunctionTok{identical}\NormalTok{(}\FunctionTok{rownames}\NormalTok{(rfy), }\FunctionTok{rownames}\NormalTok{(meta)))}

\CommentTok{\# Site centroids in CLR space}
\NormalTok{centroids }\OtherTok{\textless{}{-}} \FunctionTok{rowsum}\NormalTok{(rfy, }\AttributeTok{group =}\NormalTok{ meta}\SpecialCharTok{$}\NormalTok{site) }\SpecialCharTok{/} \FunctionTok{as.vector}\NormalTok{(}\FunctionTok{table}\NormalTok{(meta}\SpecialCharTok{$}\NormalTok{site))}

\CommentTok{\# Euclidean distance to site centroid}
\NormalTok{euclid }\OtherTok{\textless{}{-}} \ControlFlowTok{function}\NormalTok{(x, y) }\FunctionTok{sqrt}\NormalTok{(}\FunctionTok{sum}\NormalTok{((x }\SpecialCharTok{{-}}\NormalTok{ y)}\SpecialCharTok{\^{}}\DecValTok{2}\NormalTok{))}

\NormalTok{dist\_centroid }\OtherTok{\textless{}{-}} \FunctionTok{vapply}\NormalTok{(}
  \FunctionTok{seq\_len}\NormalTok{(}\FunctionTok{nrow}\NormalTok{(rfy)),}
  \ControlFlowTok{function}\NormalTok{(i) }\FunctionTok{euclid}\NormalTok{(rfy[i, ], centroids[meta}\SpecialCharTok{$}\NormalTok{site[i], ]),}
  \FunctionTok{numeric}\NormalTok{(}\DecValTok{1}\NormalTok{)}
\NormalTok{)}

\CommentTok{\# Model dataframe}
\NormalTok{df }\OtherTok{\textless{}{-}}\NormalTok{ meta}
\NormalTok{df}\SpecialCharTok{$}\NormalTok{dist\_centroid }\OtherTok{\textless{}{-}}\NormalTok{ dist\_centroid}
\NormalTok{df}\SpecialCharTok{$}\NormalTok{site }\OtherTok{\textless{}{-}} \FunctionTok{factor}\NormalTok{(df}\SpecialCharTok{$}\NormalTok{site)}
\NormalTok{df}\SpecialCharTok{$}\NormalTok{habitat }\OtherTok{\textless{}{-}} \FunctionTok{factor}\NormalTok{(df}\SpecialCharTok{$}\NormalTok{habitat, }\AttributeTok{levels =} \FunctionTok{c}\NormalTok{(}\StringTok{"forest"}\NormalTok{, }\StringTok{"subparamo"}\NormalTok{, }\StringTok{"paramo"}\NormalTok{))}

\CommentTok{\# Linear model (habitat{-}based)}
\NormalTok{mod\_hab }\OtherTok{\textless{}{-}} \FunctionTok{lm}\NormalTok{(dist\_centroid }\SpecialCharTok{\textasciitilde{}}\NormalTok{ habitat }\SpecialCharTok{*}\NormalTok{ site, }\AttributeTok{data =}\NormalTok{ df)}

\CommentTok{\# ANOVA table}
\NormalTok{a }\OtherTok{\textless{}{-}} \FunctionTok{anova}\NormalTok{(mod\_hab)}

\NormalTok{tab\_s7 }\OtherTok{\textless{}{-}} \FunctionTok{data.frame}\NormalTok{(}
  \AttributeTok{Effect =} \FunctionTok{rownames}\NormalTok{(a),}
  \AttributeTok{df =}\NormalTok{ a}\SpecialCharTok{$}\NormalTok{Df,}
  \StringTok{\textasciigrave{}}\AttributeTok{Sum of Squares}\StringTok{\textasciigrave{}} \OtherTok{=}\NormalTok{ a}\SpecialCharTok{$}\StringTok{\textasciigrave{}}\AttributeTok{Sum Sq}\StringTok{\textasciigrave{}}\NormalTok{,}
  \StringTok{\textasciigrave{}}\AttributeTok{Mean Square}\StringTok{\textasciigrave{}} \OtherTok{=}\NormalTok{ a}\SpecialCharTok{$}\StringTok{\textasciigrave{}}\AttributeTok{Mean Sq}\StringTok{\textasciigrave{}}\NormalTok{,}
  \StringTok{\textasciigrave{}}\AttributeTok{F value}\StringTok{\textasciigrave{}} \OtherTok{=}\NormalTok{ a}\SpecialCharTok{$}\StringTok{\textasciigrave{}}\AttributeTok{F value}\StringTok{\textasciigrave{}}\NormalTok{,}
  \StringTok{\textasciigrave{}}\AttributeTok{p value}\StringTok{\textasciigrave{}} \OtherTok{=}\NormalTok{ a}\SpecialCharTok{$}\StringTok{\textasciigrave{}}\AttributeTok{Pr(\textgreater{}F)}\StringTok{\textasciigrave{}}\NormalTok{,}
  \AttributeTok{row.names =} \ConstantTok{NULL}
\NormalTok{)}

\CommentTok{\# Output}
\NormalTok{out }\OtherTok{\textless{}{-}} \StringTok{"/data/lastexpansion/danieang/objects/Table\_S7\_centroid\_model.csv"}
\FunctionTok{dir.create}\NormalTok{(}\FunctionTok{dirname}\NormalTok{(out), }\AttributeTok{showWarnings =} \ConstantTok{FALSE}\NormalTok{, }\AttributeTok{recursive =} \ConstantTok{TRUE}\NormalTok{)}
\FunctionTok{write.csv}\NormalTok{(tab\_s7, out, }\AttributeTok{row.names =} \ConstantTok{FALSE}\NormalTok{)}
\end{Highlighting}
\end{Shaded}

\subsubsection{Abundance-based beta diversity
partitioning}\label{abundance-based-beta-diversity-partitioning}

Bray--Curtis dissimilarities were partitioned into balanced variation
and abundance gradients across taxonomic levels using the
\texttt{betapart} package. Metrics were calculated from rarefied
abundance data.

\begin{longtable}[]{@{}llrrr@{}}
\caption{Table S8. Bray--Curtis beta-diversity partitioning across
taxonomic levels and sites.}\tabularnewline
\toprule\noalign{}
Site & Taxonomic\_Level & beta.BRAY.BAL & beta.BRAY.GRA & beta.BRAY \\
\midrule\noalign{}
\endfirsthead
\toprule\noalign{}
Site & Taxonomic\_Level & beta.BRAY.BAL & beta.BRAY.GRA & beta.BRAY \\
\midrule\noalign{}
\endhead
\bottomrule\noalign{}
\endlastfoot
DOM & OTU & 0.9748 & 0.0244 & 0.9993 \\
DOM & Genus & 0.9164 & 0.0780 & 0.9944 \\
DOM & Order & 0.7708 & 0.2166 & 0.9874 \\
DOM & Class & 0.4256 & 0.5418 & 0.9674 \\
DOM & Phylum & 0.2759 & 0.6859 & 0.9619 \\
NV & OTU & 0.9870 & 0.0124 & 0.9995 \\
NV & Genus & 0.9553 & 0.0415 & 0.9969 \\
NV & Order & 0.7846 & 0.2059 & 0.9905 \\
NV & Class & 0.4516 & 0.5238 & 0.9754 \\
NV & Phylum & 0.1470 & 0.8046 & 0.9516 \\
BEL & OTU & 0.9870 & 0.0125 & 0.9994 \\
BEL & Genus & 0.9371 & 0.0573 & 0.9945 \\
BEL & Order & 0.7916 & 0.1995 & 0.9911 \\
BEL & Class & 0.4611 & 0.5121 & 0.9733 \\
BEL & Phylum & 0.0258 & 0.9434 & 0.9692 \\
MA & OTU & 0.9431 & 0.0557 & 0.9989 \\
MA & Genus & 0.9305 & 0.0627 & 0.9932 \\
MA & Order & 0.7966 & 0.1911 & 0.9877 \\
MA & Class & 0.5906 & 0.3750 & 0.9656 \\
MA & Phylum & 0.0000 & 0.9504 & 0.9505 \\
\end{longtable}

\subsection{Beta-regression models}\label{beta-regression-models}

To quantify elevational trends in the relative abundance of dominant
fungal lineages, we fitted beta-regression models to order-level
relative read abundances. Models were fitted separately for each order
using a logit link, with elevation as a continuous predictor and site
included as a fixed effect. Relative abundances were calculated as
proportions of total reads per sample and transformed using a
Smithson--Verkuilen adjustment.

We focus on helotiales and sebacinales, the two most abundant orders,
which also showed the only significant elevation responses. For both
orders, fitted relationships and raw data are shown in Figure 2, while
full model summaries are provided in Table S9. Elevation effects for the
eight most abundant orders are summarized in Table S10.

Model diagnostics were examined to assess fit and identify potential
violations of distributional assumptions. Diagnostic plots include
Pearson residuals versus fitted values and elevation, quantile residual
Q--Q plots, and leverage values (Figures S6--S7).

\textbf{Table S9}:

\begin{longtable}[]{@{}llrrrr@{}}
\caption{Table S9a. Beta regression summary for helotiales (mean model
coefficients and precision (phi)).}\tabularnewline
\toprule\noalign{}
Model & Term & Estimate & SE & z & p \\
\midrule\noalign{}
\endfirsthead
\toprule\noalign{}
Model & Term & Estimate & SE & z & p \\
\midrule\noalign{}
\endhead
\bottomrule\noalign{}
\endlastfoot
helotiales & (Intercept) & -3.2040 & 1.2559 & -2.5512 & 0.0107 \\
helotiales & elevation & 0.0010 & 0.0004 & 2.3959 & 0.0166 \\
helotiales & siteDOM & -0.7291 & 0.3651 & -1.9969 & 0.0458 \\
helotiales & siteMA & 0.0050 & 0.3871 & 0.0130 & 0.9896 \\
helotiales & siteNV & -0.8864 & 0.3346 & -2.6494 & 0.0081 \\
helotiales & (phi) (phi) & 6.2698 & 1.0711 & 5.8534 & 0.0000 \\
\end{longtable}

\begin{longtable}[]{@{}llrrrr@{}}
\caption{Table S9b. Beta regression summary for sebacinales (mean model
coefficients and precision (phi)).}\tabularnewline
\toprule\noalign{}
Model & Term & Estimate & SE & z & p \\
\midrule\noalign{}
\endfirsthead
\toprule\noalign{}
Model & Term & Estimate & SE & z & p \\
\midrule\noalign{}
\endhead
\bottomrule\noalign{}
\endlastfoot
sebacinales & (Intercept) & 0.4169 & 1.3941 & 0.2990 & 0.7649 \\
sebacinales & elevation & -0.0010 & 0.0005 & -2.1320 & 0.0330 \\
sebacinales & siteDOM & 0.7196 & 0.4162 & 1.7289 & 0.0838 \\
sebacinales & siteMA & 0.3029 & 0.4603 & 0.6580 & 0.5105 \\
sebacinales & siteNV & 0.9779 & 0.3714 & 2.6333 & 0.0085 \\
sebacinales & (phi) (phi) & 10.1810 & 2.0241 & 5.0298 & 0.0000 \\
\end{longtable}

\begin{center}\rule{0.5\linewidth}{0.5pt}\end{center}

\textbf{Table S10}:

\begin{longtable}[]{@{}
  >{\raggedright\arraybackslash}p{(\columnwidth - 12\tabcolsep) * \real{0.2388}}
  >{\raggedright\arraybackslash}p{(\columnwidth - 12\tabcolsep) * \real{0.1493}}
  >{\raggedleft\arraybackslash}p{(\columnwidth - 12\tabcolsep) * \real{0.1343}}
  >{\raggedleft\arraybackslash}p{(\columnwidth - 12\tabcolsep) * \real{0.0896}}
  >{\raggedleft\arraybackslash}p{(\columnwidth - 12\tabcolsep) * \real{0.1194}}
  >{\raggedleft\arraybackslash}p{(\columnwidth - 12\tabcolsep) * \real{0.1194}}
  >{\raggedleft\arraybackslash}p{(\columnwidth - 12\tabcolsep) * \real{0.1493}}@{}}
\caption{Table S10. Elevation effects from beta regression models for
the dominant orders (mean submodel), with pseudo-R².}\tabularnewline
\toprule\noalign{}
\begin{minipage}[b]{\linewidth}\raggedright
order
\end{minipage} & \begin{minipage}[b]{\linewidth}\raggedright
term
\end{minipage} & \begin{minipage}[b]{\linewidth}\raggedleft
estimate
\end{minipage} & \begin{minipage}[b]{\linewidth}\raggedleft
se
\end{minipage} & \begin{minipage}[b]{\linewidth}\raggedleft
z
\end{minipage} & \begin{minipage}[b]{\linewidth}\raggedleft
p\_value
\end{minipage} & \begin{minipage}[b]{\linewidth}\raggedleft
pseudo\_R2
\end{minipage} \\
\midrule\noalign{}
\endfirsthead
\toprule\noalign{}
\begin{minipage}[b]{\linewidth}\raggedright
order
\end{minipage} & \begin{minipage}[b]{\linewidth}\raggedright
term
\end{minipage} & \begin{minipage}[b]{\linewidth}\raggedleft
estimate
\end{minipage} & \begin{minipage}[b]{\linewidth}\raggedleft
se
\end{minipage} & \begin{minipage}[b]{\linewidth}\raggedleft
z
\end{minipage} & \begin{minipage}[b]{\linewidth}\raggedleft
p\_value
\end{minipage} & \begin{minipage}[b]{\linewidth}\raggedleft
pseudo\_R2
\end{minipage} \\
\midrule\noalign{}
\endhead
\bottomrule\noalign{}
\endlastfoot
helotiales & elevation & 1e-03 & 4e-04 & 2.3959 & 0.0166 & 0.2133 \\
sebacinales & elevation & -1e-03 & 5e-04 & -2.1320 & 0.0330 & 0.2261 \\
chaetothyriales & elevation & 1e-04 & 4e-04 & 0.2327 & 0.8160 &
0.0239 \\
sclerococcales & elevation & 7e-04 & 5e-04 & 1.3758 & 0.1689 & 0.1330 \\
agaricales & elevation & -7e-04 & 4e-04 & -1.7597 & 0.0785 & 0.0929 \\
pleosporales & elevation & -4e-04 & 4e-04 & -1.0668 & 0.2861 & 0.1528 \\
hypocreales & elevation & -2e-04 & 3e-04 & -0.7300 & 0.4654 & 0.2397 \\
leotiales & elevation & -2e-04 & 3e-04 & -0.6812 & 0.4957 & 0.2964 \\
\end{longtable}

\begin{center}\rule{0.5\linewidth}{0.5pt}\end{center}

\includegraphics{figures/Fig_S7_betareg_diagnostics_helotiales.png}

\textbf{Figure S6}: Diagnostic plots for the beta-regression model of
Helotiales. \emph{Code: see ``Beta-regression models'' below.}

\begin{center}\rule{0.5\linewidth}{0.5pt}\end{center}

\includegraphics{figures/Fig_S8_betareg_diagnostics_sebacinales.png}

\textbf{Figure S7}: Diagnostic plots for the beta-regression model of
Sebacinales. \emph{Code: see ``Beta-regression models'' below.}

\begin{Shaded}
\begin{Highlighting}[]
\CommentTok{\# This code fits beta{-}regression models for dominant fungal orders,}
\CommentTok{\# generates Tables S9–S10, and exports Figures 2a and S6–S7 as PNG files.}

\FunctionTok{library}\NormalTok{(phyloseq)}
\FunctionTok{library}\NormalTok{(dplyr)}
\FunctionTok{library}\NormalTok{(tidyr)}
\FunctionTok{library}\NormalTok{(ggplot2)}
\FunctionTok{library}\NormalTok{(betareg)}
\FunctionTok{library}\NormalTok{(patchwork)}


\NormalTok{out\_fig\_dir }\OtherTok{\textless{}{-}} \StringTok{"/data/lastexpansion/danieang/objects"} 
\NormalTok{out\_tab\_dir }\OtherTok{\textless{}{-}} \StringTok{"/data/lastexpansion/danieang/objects"} 

\FunctionTok{set.seed}\NormalTok{(}\DecValTok{1}\NormalTok{)}

\DocumentationTok{\#\# {-}{-}{-}{-} Helper: build beta{-}regression dataframe for an Order {-}{-}{-}{-}}
\NormalTok{make\_beta\_df\_from\_order }\OtherTok{\textless{}{-}} \ControlFlowTok{function}\NormalTok{(ps, order\_name) \{}

\NormalTok{  otu }\OtherTok{\textless{}{-}} \FunctionTok{as}\NormalTok{(}\FunctionTok{otu\_table}\NormalTok{(ps), }\StringTok{"matrix"}\NormalTok{)}
  \ControlFlowTok{if}\NormalTok{ (}\SpecialCharTok{!}\FunctionTok{taxa\_are\_rows}\NormalTok{(ps)) otu }\OtherTok{\textless{}{-}} \FunctionTok{t}\NormalTok{(otu)  }\CommentTok{\# taxa x samples}

\NormalTok{  sd }\OtherTok{\textless{}{-}} \FunctionTok{as.data.frame}\NormalTok{(}\FunctionTok{sample\_data}\NormalTok{(ps))}
\NormalTok{  sd }\OtherTok{\textless{}{-}}\NormalTok{ sd[}\FunctionTok{colnames}\NormalTok{(otu), , drop }\OtherTok{=} \ConstantTok{FALSE}\NormalTok{]  }\CommentTok{\# force alignment}

\NormalTok{  tax }\OtherTok{\textless{}{-}} \FunctionTok{as.data.frame}\NormalTok{(}\FunctionTok{tax\_table}\NormalTok{(ps), }\AttributeTok{stringsAsFactors =} \ConstantTok{FALSE}\NormalTok{)}
  \FunctionTok{stopifnot}\NormalTok{(}\StringTok{"Order"} \SpecialCharTok{\%in\%} \FunctionTok{names}\NormalTok{(tax))}

\NormalTok{  order\_vec }\OtherTok{\textless{}{-}} \FunctionTok{tolower}\NormalTok{(}\FunctionTok{as.character}\NormalTok{(tax}\SpecialCharTok{$}\NormalTok{Order))}
\NormalTok{  order\_vec[}\FunctionTok{is.na}\NormalTok{(order\_vec)] }\OtherTok{\textless{}{-}} \StringTok{""}

\NormalTok{  otu\_ids }\OtherTok{\textless{}{-}} \FunctionTok{rownames}\NormalTok{(tax)[}\FunctionTok{grepl}\NormalTok{(}\FunctionTok{tolower}\NormalTok{(order\_name), order\_vec, }\AttributeTok{fixed =} \ConstantTok{TRUE}\NormalTok{)]}
  \ControlFlowTok{if}\NormalTok{ (}\FunctionTok{length}\NormalTok{(otu\_ids) }\SpecialCharTok{==} \DecValTok{0}\NormalTok{) }\FunctionTok{stop}\NormalTok{(}\FunctionTok{paste}\NormalTok{(}\StringTok{"No OTUs found for"}\NormalTok{, order\_name))}

\NormalTok{  counts }\OtherTok{\textless{}{-}} \FunctionTok{colSums}\NormalTok{(otu[otu\_ids, , }\AttributeTok{drop =} \ConstantTok{FALSE}\NormalTok{])}
\NormalTok{  total  }\OtherTok{\textless{}{-}} \FunctionTok{colSums}\NormalTok{(otu)}

\NormalTok{  samps }\OtherTok{\textless{}{-}} \FunctionTok{colnames}\NormalTok{(otu)}

\NormalTok{  df }\OtherTok{\textless{}{-}} \FunctionTok{data.frame}\NormalTok{(}
    \AttributeTok{sample    =}\NormalTok{ samps,}
    \AttributeTok{count     =} \FunctionTok{as.numeric}\NormalTok{(counts),}
    \AttributeTok{total     =} \FunctionTok{as.numeric}\NormalTok{(total),}
    \AttributeTok{prop\_raw  =} \FunctionTok{ifelse}\NormalTok{(total }\SpecialCharTok{\textgreater{}} \DecValTok{0}\NormalTok{, counts }\SpecialCharTok{/}\NormalTok{ total, }\ConstantTok{NA\_real\_}\NormalTok{),}
    \AttributeTok{site      =}\NormalTok{ sd[, }\StringTok{"site"}\NormalTok{, }\AttributeTok{drop =} \ConstantTok{TRUE}\NormalTok{],}
    \AttributeTok{elevation =}\NormalTok{ sd[, }\StringTok{"elevation"}\NormalTok{, }\AttributeTok{drop =} \ConstantTok{TRUE}\NormalTok{],}
    \AttributeTok{stringsAsFactors =} \ConstantTok{FALSE}
\NormalTok{  )}

  \CommentTok{\# Smithson–Verkuilen adjustment to (0,1)}
\NormalTok{  n }\OtherTok{\textless{}{-}} \FunctionTok{nrow}\NormalTok{(df)}
\NormalTok{  df}\SpecialCharTok{$}\NormalTok{Proportion }\OtherTok{\textless{}{-}}\NormalTok{ (df}\SpecialCharTok{$}\NormalTok{prop\_raw }\SpecialCharTok{*}\NormalTok{ (n }\SpecialCharTok{{-}} \DecValTok{1}\NormalTok{) }\SpecialCharTok{+} \FloatTok{0.5}\NormalTok{) }\SpecialCharTok{/}\NormalTok{ n}

\NormalTok{  df}\SpecialCharTok{$}\NormalTok{site }\OtherTok{\textless{}{-}} \FunctionTok{factor}\NormalTok{(df}\SpecialCharTok{$}\NormalTok{site)}
\NormalTok{  df}\SpecialCharTok{$}\NormalTok{elevation }\OtherTok{\textless{}{-}} \FunctionTok{as.numeric}\NormalTok{(df}\SpecialCharTok{$}\NormalTok{elevation)}

\NormalTok{  df}
\NormalTok{\}}

\DocumentationTok{\#\# {-}{-}{-}{-} Helper: tidy coefficient table for Table S8 {-}{-}{-}{-}}
\NormalTok{tidy\_betareg\_mean }\OtherTok{\textless{}{-}} \ControlFlowTok{function}\NormalTok{(mod, model\_name) \{}
\NormalTok{  sm }\OtherTok{\textless{}{-}} \FunctionTok{summary}\NormalTok{(mod)}
\NormalTok{  co }\OtherTok{\textless{}{-}} \FunctionTok{as.data.frame}\NormalTok{(sm}\SpecialCharTok{$}\NormalTok{coefficients}\SpecialCharTok{$}\NormalTok{mean)}
\NormalTok{  co}\SpecialCharTok{$}\NormalTok{Term }\OtherTok{\textless{}{-}} \FunctionTok{rownames}\NormalTok{(co)}
  \FunctionTok{rownames}\NormalTok{(co) }\OtherTok{\textless{}{-}} \ConstantTok{NULL}
  \FunctionTok{names}\NormalTok{(co) }\OtherTok{\textless{}{-}} \FunctionTok{c}\NormalTok{(}\StringTok{"Estimate"}\NormalTok{,}\StringTok{"SE"}\NormalTok{,}\StringTok{"z"}\NormalTok{,}\StringTok{"p"}\NormalTok{,}\StringTok{"Term"}\NormalTok{)}

  \CommentTok{\# move Term first + add model label}
\NormalTok{  co }\OtherTok{\textless{}{-}}\NormalTok{ co }\SpecialCharTok{\%\textgreater{}\%}
    \FunctionTok{select}\NormalTok{(Term, Estimate, SE, z, p) }\SpecialCharTok{\%\textgreater{}\%}
    \FunctionTok{mutate}\NormalTok{(}\AttributeTok{Model =}\NormalTok{ model\_name, }\AttributeTok{.before =} \DecValTok{1}\NormalTok{)}

  \CommentTok{\# also add phi (precision) row, for completeness}
\NormalTok{  phi }\OtherTok{\textless{}{-}} \FunctionTok{as.data.frame}\NormalTok{(sm}\SpecialCharTok{$}\NormalTok{coefficients}\SpecialCharTok{$}\NormalTok{precision)}
\NormalTok{  phi}\SpecialCharTok{$}\NormalTok{Term }\OtherTok{\textless{}{-}} \FunctionTok{rownames}\NormalTok{(phi); }\FunctionTok{rownames}\NormalTok{(phi) }\OtherTok{\textless{}{-}} \ConstantTok{NULL}
  \FunctionTok{names}\NormalTok{(phi) }\OtherTok{\textless{}{-}} \FunctionTok{c}\NormalTok{(}\StringTok{"Estimate"}\NormalTok{,}\StringTok{"SE"}\NormalTok{,}\StringTok{"z"}\NormalTok{,}\StringTok{"p"}\NormalTok{,}\StringTok{"Term"}\NormalTok{)}
\NormalTok{  phi }\OtherTok{\textless{}{-}}\NormalTok{ phi }\SpecialCharTok{\%\textgreater{}\%}
    \FunctionTok{select}\NormalTok{(Term, Estimate, SE, z, p) }\SpecialCharTok{\%\textgreater{}\%}
    \FunctionTok{mutate}\NormalTok{(}\AttributeTok{Model =}\NormalTok{ model\_name, }\AttributeTok{.before =} \DecValTok{1}\NormalTok{) }\SpecialCharTok{\%\textgreater{}\%}
    \FunctionTok{mutate}\NormalTok{(}\AttributeTok{Term =} \FunctionTok{paste0}\NormalTok{(Term, }\StringTok{" (phi)"}\NormalTok{))}

  \FunctionTok{bind\_rows}\NormalTok{(co, phi)}
\NormalTok{\}}

\DocumentationTok{\#\# {-}{-}{-}{-} Helper: elevation{-}only summary across orders for Table S9 {-}{-}{-}{-}}
\NormalTok{extract\_elev }\OtherTok{\textless{}{-}} \ControlFlowTok{function}\NormalTok{(mod, order\_name) \{}
\NormalTok{  sm }\OtherTok{\textless{}{-}} \FunctionTok{summary}\NormalTok{(mod)}
\NormalTok{  co }\OtherTok{\textless{}{-}}\NormalTok{ sm}\SpecialCharTok{$}\NormalTok{coefficients}\SpecialCharTok{$}\NormalTok{mean}
  \ControlFlowTok{if}\NormalTok{ (}\SpecialCharTok{!}\NormalTok{(}\StringTok{"elevation"} \SpecialCharTok{\%in\%} \FunctionTok{rownames}\NormalTok{(co))) }\FunctionTok{return}\NormalTok{(}\ConstantTok{NULL}\NormalTok{)}
  \FunctionTok{data.frame}\NormalTok{(}
    \AttributeTok{order =}\NormalTok{ order\_name,}
    \AttributeTok{term  =} \StringTok{"elevation"}\NormalTok{,}
    \AttributeTok{estimate  =}\NormalTok{ co[}\StringTok{"elevation"}\NormalTok{,}\StringTok{"Estimate"}\NormalTok{],}
    \AttributeTok{se        =}\NormalTok{ co[}\StringTok{"elevation"}\NormalTok{,}\StringTok{"Std. Error"}\NormalTok{],}
    \AttributeTok{z         =}\NormalTok{ co[}\StringTok{"elevation"}\NormalTok{,}\StringTok{"z value"}\NormalTok{],}
    \AttributeTok{p\_value   =}\NormalTok{ co[}\StringTok{"elevation"}\NormalTok{,}\StringTok{"Pr(\textgreater{}|z|)"}\NormalTok{],}
    \AttributeTok{pseudo\_R2 =}\NormalTok{ sm}\SpecialCharTok{$}\NormalTok{pseudo.r.squared,}
    \AttributeTok{stringsAsFactors =} \ConstantTok{FALSE}
\NormalTok{  )}
\NormalTok{\}}

\DocumentationTok{\#\# {-}{-}{-}{-} 1) Fit models {-}{-}{-}{-}}
\NormalTok{orders\_target }\OtherTok{\textless{}{-}} \FunctionTok{c}\NormalTok{(}
  \StringTok{"helotiales"}\NormalTok{,}
  \StringTok{"sebacinales"}\NormalTok{,}
  \StringTok{"chaetothyriales"}\NormalTok{,}
  \StringTok{"sclerococcales"}\NormalTok{,}
  \StringTok{"agaricales"}\NormalTok{,}
  \StringTok{"pleosporales"}\NormalTok{,}
  \StringTok{"hypocreales"}\NormalTok{,}
  \StringTok{"leotiales"}
\NormalTok{)}

\NormalTok{df\_list  }\OtherTok{\textless{}{-}} \FunctionTok{setNames}\NormalTok{(}\FunctionTok{vector}\NormalTok{(}\StringTok{"list"}\NormalTok{, }\FunctionTok{length}\NormalTok{(orders\_target)), orders\_target)}
\NormalTok{mod\_list }\OtherTok{\textless{}{-}} \FunctionTok{setNames}\NormalTok{(}\FunctionTok{vector}\NormalTok{(}\StringTok{"list"}\NormalTok{, }\FunctionTok{length}\NormalTok{(orders\_target)), orders\_target)}

\ControlFlowTok{for}\NormalTok{ (ord }\ControlFlowTok{in}\NormalTok{ orders\_target) \{}
  \FunctionTok{message}\NormalTok{(}\StringTok{"Building DF + fitting: "}\NormalTok{, ord)}
\NormalTok{  df\_list[[ord]]  }\OtherTok{\textless{}{-}} \FunctionTok{make\_beta\_df\_from\_order}\NormalTok{(ps, ord)}
\NormalTok{  mod\_list[[ord]] }\OtherTok{\textless{}{-}} \FunctionTok{betareg}\NormalTok{(Proportion }\SpecialCharTok{\textasciitilde{}}\NormalTok{ elevation }\SpecialCharTok{+}\NormalTok{ site, }\AttributeTok{data =}\NormalTok{ df\_list[[ord]])}
\NormalTok{\}}

\DocumentationTok{\#\# {-}{-}{-}{-} 2) Table S9: helotiales + sebacinales full model tables {-}{-}{-}{-}}
\NormalTok{tab\_s9\_helot }\OtherTok{\textless{}{-}} \FunctionTok{tidy\_betareg\_mean}\NormalTok{(mod\_list[[}\StringTok{"helotiales"}\NormalTok{]], }\StringTok{"helotiales"}\NormalTok{)}
\NormalTok{tab\_s9\_seba  }\OtherTok{\textless{}{-}} \FunctionTok{tidy\_betareg\_mean}\NormalTok{(mod\_list[[}\StringTok{"sebacinales"}\NormalTok{]], }\StringTok{"sebacinales"}\NormalTok{)}

\FunctionTok{write.csv}\NormalTok{(tab\_s9\_helot,}
          \FunctionTok{file.path}\NormalTok{(out\_tab\_dir, }\StringTok{"Table\_S9\_betareg\_helotiales.csv"}\NormalTok{),}
          \AttributeTok{row.names =} \ConstantTok{FALSE}\NormalTok{)}
\FunctionTok{write.csv}\NormalTok{(tab\_s9\_seba,}
          \FunctionTok{file.path}\NormalTok{(out\_tab\_dir, }\StringTok{"Table\_S9\_betareg\_sebacinales.csv"}\NormalTok{),}
          \AttributeTok{row.names =} \ConstantTok{FALSE}\NormalTok{)}

\DocumentationTok{\#\# {-}{-}{-}{-} 3) Table S10: elevation effects across orders {-}{-}{-}{-}}
\NormalTok{tab\_s10 }\OtherTok{\textless{}{-}} \FunctionTok{bind\_rows}\NormalTok{(}\FunctionTok{lapply}\NormalTok{(}\FunctionTok{names}\NormalTok{(mod\_list), \textbackslash{}(ord) }\FunctionTok{extract\_elev}\NormalTok{(mod\_list[[ord]], ord)))}
\FunctionTok{write.csv}\NormalTok{(tab\_s10,}
          \FunctionTok{file.path}\NormalTok{(out\_tab\_dir, }\StringTok{"Table\_S10\_betareg\_elevation\_effects.csv"}\NormalTok{),}
          \AttributeTok{row.names =} \ConstantTok{FALSE}\NormalTok{)}



\DocumentationTok{\#\# {-}{-}{-}{-} 4) Beta regression plots for Helotiales + Sebacinales {-}{-}{-}{-}}

\CommentTok{\# helper}
\NormalTok{inv\_logit }\OtherTok{\textless{}{-}} \ControlFlowTok{function}\NormalTok{(x) }\DecValTok{1} \SpecialCharTok{/}\NormalTok{ (}\DecValTok{1} \SpecialCharTok{+} \FunctionTok{exp}\NormalTok{(}\SpecialCharTok{{-}}\NormalTok{x))}

\NormalTok{plot\_betareg\_with\_raw }\OtherTok{\textless{}{-}} \ControlFlowTok{function}\NormalTok{(model, data, }\AttributeTok{label\_prefix =} \StringTok{"helotiales"}\NormalTok{) \{}

  \CommentTok{\# Ensure required columns exist}
  \FunctionTok{stopifnot}\NormalTok{(}\FunctionTok{all}\NormalTok{(}\FunctionTok{c}\NormalTok{(}\StringTok{"site"}\NormalTok{,}\StringTok{"elevation"}\NormalTok{,}\StringTok{"prop\_raw"}\NormalTok{) }\SpecialCharTok{\%in\%} \FunctionTok{names}\NormalTok{(data)))}

\NormalTok{  data }\OtherTok{\textless{}{-}}\NormalTok{ data[}\FunctionTok{complete.cases}\NormalTok{(data[, }\FunctionTok{c}\NormalTok{(}\StringTok{"site"}\NormalTok{,}\StringTok{"elevation"}\NormalTok{,}\StringTok{"prop\_raw"}\NormalTok{)]), , drop }\OtherTok{=} \ConstantTok{FALSE}\NormalTok{]}

  \CommentTok{\# Align site factor levels to model}
  \ControlFlowTok{if}\NormalTok{ (}\SpecialCharTok{!}\FunctionTok{is.null}\NormalTok{(model}\SpecialCharTok{$}\NormalTok{xlevels}\SpecialCharTok{$}\NormalTok{mean}\SpecialCharTok{$}\NormalTok{site)) \{}
\NormalTok{    data}\SpecialCharTok{$}\NormalTok{site }\OtherTok{\textless{}{-}} \FunctionTok{factor}\NormalTok{(}\FunctionTok{as.character}\NormalTok{(data}\SpecialCharTok{$}\NormalTok{site), }\AttributeTok{levels =}\NormalTok{ model}\SpecialCharTok{$}\NormalTok{xlevels}\SpecialCharTok{$}\NormalTok{mean}\SpecialCharTok{$}\NormalTok{site)}
\NormalTok{  \} }\ControlFlowTok{else}\NormalTok{ \{}
\NormalTok{    data}\SpecialCharTok{$}\NormalTok{site }\OtherTok{\textless{}{-}} \FunctionTok{factor}\NormalTok{(data}\SpecialCharTok{$}\NormalTok{site)}
\NormalTok{  \}}

  \CommentTok{\# Drop any rows that became NA due to level mismatch}
\NormalTok{  data }\OtherTok{\textless{}{-}}\NormalTok{ data[}\SpecialCharTok{!}\FunctionTok{is.na}\NormalTok{(data}\SpecialCharTok{$}\NormalTok{site), , drop }\OtherTok{=} \ConstantTok{FALSE}\NormalTok{]}

\NormalTok{  one\_level\_site }\OtherTok{\textless{}{-}}\NormalTok{ (}\FunctionTok{nlevels}\NormalTok{(}\FunctionTok{droplevels}\NormalTok{(data}\SpecialCharTok{$}\NormalTok{site)) }\SpecialCharTok{\textless{}} \DecValTok{2}\NormalTok{)}

  \CommentTok{\# Build prediction grid by site across observed elevation range}
\NormalTok{  newdat }\OtherTok{\textless{}{-}}\NormalTok{ data }\SpecialCharTok{\%\textgreater{}\%}
\NormalTok{    dplyr}\SpecialCharTok{::}\FunctionTok{select}\NormalTok{(site, elevation) }\SpecialCharTok{\%\textgreater{}\%}
\NormalTok{    dplyr}\SpecialCharTok{::}\FunctionTok{distinct}\NormalTok{() }\SpecialCharTok{\%\textgreater{}\%}
\NormalTok{    dplyr}\SpecialCharTok{::}\FunctionTok{group\_by}\NormalTok{(site) }\SpecialCharTok{\%\textgreater{}\%}
\NormalTok{    dplyr}\SpecialCharTok{::}\FunctionTok{summarize}\NormalTok{(}
      \AttributeTok{elev\_min =} \FunctionTok{min}\NormalTok{(elevation, }\AttributeTok{na.rm =} \ConstantTok{TRUE}\NormalTok{),}
      \AttributeTok{elev\_max =} \FunctionTok{max}\NormalTok{(elevation, }\AttributeTok{na.rm =} \ConstantTok{TRUE}\NormalTok{),}
      \AttributeTok{.groups =} \StringTok{"drop"}
\NormalTok{    ) }\SpecialCharTok{\%\textgreater{}\%}
\NormalTok{    dplyr}\SpecialCharTok{::}\FunctionTok{rowwise}\NormalTok{() }\SpecialCharTok{\%\textgreater{}\%}
\NormalTok{    dplyr}\SpecialCharTok{::}\FunctionTok{mutate}\NormalTok{(}\AttributeTok{elevation =} \FunctionTok{list}\NormalTok{(}\FunctionTok{seq}\NormalTok{(elev\_min, elev\_max, }\AttributeTok{length.out =} \DecValTok{100}\NormalTok{))) }\SpecialCharTok{\%\textgreater{}\%}
\NormalTok{    tidyr}\SpecialCharTok{::}\FunctionTok{unnest}\NormalTok{(elevation) }\SpecialCharTok{\%\textgreater{}\%}
\NormalTok{    dplyr}\SpecialCharTok{::}\FunctionTok{select}\NormalTok{(site, elevation) }\SpecialCharTok{\%\textgreater{}\%}
\NormalTok{    dplyr}\SpecialCharTok{::}\FunctionTok{ungroup}\NormalTok{()}

  \CommentTok{\# Re{-}apply model levels to newdat site}
  \ControlFlowTok{if}\NormalTok{ (}\SpecialCharTok{!}\FunctionTok{is.null}\NormalTok{(model}\SpecialCharTok{$}\NormalTok{xlevels}\SpecialCharTok{$}\NormalTok{mean}\SpecialCharTok{$}\NormalTok{site)) \{}
\NormalTok{    newdat}\SpecialCharTok{$}\NormalTok{site }\OtherTok{\textless{}{-}} \FunctionTok{factor}\NormalTok{(}\FunctionTok{as.character}\NormalTok{(newdat}\SpecialCharTok{$}\NormalTok{site), }\AttributeTok{levels =}\NormalTok{ model}\SpecialCharTok{$}\NormalTok{xlevels}\SpecialCharTok{$}\NormalTok{mean}\SpecialCharTok{$}\NormalTok{site)}
\NormalTok{  \} }\ControlFlowTok{else}\NormalTok{ \{}
\NormalTok{    newdat}\SpecialCharTok{$}\NormalTok{site }\OtherTok{\textless{}{-}} \FunctionTok{factor}\NormalTok{(newdat}\SpecialCharTok{$}\NormalTok{site)}
\NormalTok{  \}}

  \CommentTok{\# If we only have 1 site level, disable contrasts to prevent the error}
  \ControlFlowTok{if}\NormalTok{ (one\_level\_site) \{}
    \FunctionTok{contrasts}\NormalTok{(newdat}\SpecialCharTok{$}\NormalTok{site) }\OtherTok{\textless{}{-}} \ConstantTok{NULL}
\NormalTok{  \}}

  \CommentTok{\# Design matrix for mean submodel}
\NormalTok{  mean\_terms }\OtherTok{\textless{}{-}}\NormalTok{ stats}\SpecialCharTok{::}\FunctionTok{delete.response}\NormalTok{(stats}\SpecialCharTok{::}\FunctionTok{terms}\NormalTok{(model, }\AttributeTok{model =} \StringTok{"mean"}\NormalTok{))}
\NormalTok{  X }\OtherTok{\textless{}{-}} \FunctionTok{model.matrix}\NormalTok{(mean\_terms, newdat)}

\NormalTok{  beta }\OtherTok{\textless{}{-}} \FunctionTok{coef}\NormalTok{(model, }\AttributeTok{model =} \StringTok{"mean"}\NormalTok{)}
\NormalTok{  V }\OtherTok{\textless{}{-}} \FunctionTok{try}\NormalTok{(}\FunctionTok{vcov}\NormalTok{(model, }\AttributeTok{model =} \StringTok{"mean"}\NormalTok{), }\AttributeTok{silent =} \ConstantTok{TRUE}\NormalTok{)}
  \ControlFlowTok{if}\NormalTok{ (}\FunctionTok{inherits}\NormalTok{(V, }\StringTok{"try{-}error"}\NormalTok{)) V }\OtherTok{\textless{}{-}} \FunctionTok{vcov}\NormalTok{(model)}

\NormalTok{  eta    }\OtherTok{\textless{}{-}} \FunctionTok{as.vector}\NormalTok{(X }\SpecialCharTok{\%*\%}\NormalTok{ beta)}
\NormalTok{  se\_eta }\OtherTok{\textless{}{-}} \FunctionTok{sqrt}\NormalTok{(}\FunctionTok{rowSums}\NormalTok{((X }\SpecialCharTok{\%*\%}\NormalTok{ V) }\SpecialCharTok{*}\NormalTok{ X))}

\NormalTok{  newdat}\SpecialCharTok{$}\NormalTok{fit }\OtherTok{\textless{}{-}} \FunctionTok{inv\_logit}\NormalTok{(eta)}
\NormalTok{  newdat}\SpecialCharTok{$}\NormalTok{lwr }\OtherTok{\textless{}{-}} \FunctionTok{inv\_logit}\NormalTok{(eta }\SpecialCharTok{{-}} \FloatTok{1.96} \SpecialCharTok{*}\NormalTok{ se\_eta)}
\NormalTok{  newdat}\SpecialCharTok{$}\NormalTok{upr }\OtherTok{\textless{}{-}} \FunctionTok{inv\_logit}\NormalTok{(eta }\SpecialCharTok{+} \FloatTok{1.96} \SpecialCharTok{*}\NormalTok{ se\_eta)}

  \CommentTok{\# Plot}
  \FunctionTok{ggplot}\NormalTok{() }\SpecialCharTok{+}
    \FunctionTok{geom\_point}\NormalTok{(}
      \AttributeTok{data =}\NormalTok{ data,}
      \FunctionTok{aes}\NormalTok{(}\AttributeTok{x =}\NormalTok{ elevation, }\AttributeTok{y =}\NormalTok{ prop\_raw, }\AttributeTok{colour =}\NormalTok{ site),}
      \AttributeTok{size =} \DecValTok{2}\NormalTok{, }\AttributeTok{alpha =} \FloatTok{0.7}
\NormalTok{    ) }\SpecialCharTok{+}
    \FunctionTok{geom\_ribbon}\NormalTok{(}
      \AttributeTok{data =}\NormalTok{ newdat,}
      \FunctionTok{aes}\NormalTok{(}\AttributeTok{x =}\NormalTok{ elevation, }\AttributeTok{ymin =}\NormalTok{ lwr, }\AttributeTok{ymax =}\NormalTok{ upr, }\AttributeTok{fill =}\NormalTok{ site, }\AttributeTok{group =}\NormalTok{ site),}
      \AttributeTok{alpha =} \FloatTok{0.15}\NormalTok{, }\AttributeTok{colour =} \ConstantTok{NA}
\NormalTok{    ) }\SpecialCharTok{+}
    \FunctionTok{geom\_line}\NormalTok{(}
      \AttributeTok{data =}\NormalTok{ newdat,}
      \FunctionTok{aes}\NormalTok{(}\AttributeTok{x =}\NormalTok{ elevation, }\AttributeTok{y =}\NormalTok{ fit, }\AttributeTok{colour =}\NormalTok{ site, }\AttributeTok{group =}\NormalTok{ site),}
      \AttributeTok{linewidth =} \FloatTok{0.8}
\NormalTok{    ) }\SpecialCharTok{+}
    \FunctionTok{scale\_y\_continuous}\NormalTok{(}\StringTok{"Relative abundance (raw proportion)"}\NormalTok{, }\AttributeTok{limits =} \FunctionTok{c}\NormalTok{(}\DecValTok{0}\NormalTok{, }\DecValTok{1}\NormalTok{)) }\SpecialCharTok{+}
    \FunctionTok{scale\_x\_continuous}\NormalTok{(}\StringTok{"Elevation (m)"}\NormalTok{) }\SpecialCharTok{+}
    \FunctionTok{ggtitle}\NormalTok{(label\_prefix) }\SpecialCharTok{+}
    \FunctionTok{theme\_bw}\NormalTok{(}\AttributeTok{base\_size =} \DecValTok{10}\NormalTok{) }\SpecialCharTok{+}
    \FunctionTok{theme}\NormalTok{(}
      \AttributeTok{plot.title =} \FunctionTok{element\_text}\NormalTok{(}\AttributeTok{face =} \StringTok{"bold"}\NormalTok{, }\AttributeTok{size =} \DecValTok{12}\NormalTok{, }\AttributeTok{hjust =} \DecValTok{0}\NormalTok{),}
      \AttributeTok{legend.position =} \StringTok{"bottom"}
\NormalTok{    )}
\NormalTok{\}}

\CommentTok{\# Build plots}
\NormalTok{p\_helot }\OtherTok{\textless{}{-}} \FunctionTok{plot\_betareg\_with\_raw}\NormalTok{(mod\_list[[}\StringTok{"helotiales"}\NormalTok{]], df\_list[[}\StringTok{"helotiales"}\NormalTok{]], }\StringTok{"A) helotiales"}\NormalTok{)}
\NormalTok{p\_seba  }\OtherTok{\textless{}{-}} \FunctionTok{plot\_betareg\_with\_raw}\NormalTok{(mod\_list[[}\StringTok{"sebacinales"}\NormalTok{]], df\_list[[}\StringTok{"sebacinales"}\NormalTok{]], }\StringTok{"B) sebacinales"}\NormalTok{)}

\CommentTok{\# Combine and save}
\NormalTok{panel\_beta }\OtherTok{\textless{}{-}}\NormalTok{ (p\_helot }\SpecialCharTok{|}\NormalTok{ p\_seba) }\SpecialCharTok{+}\NormalTok{ patchwork}\SpecialCharTok{::}\FunctionTok{plot\_layout}\NormalTok{(}\AttributeTok{guides =} \StringTok{"collect"}\NormalTok{)}

\FunctionTok{ggsave}\NormalTok{(}
  \AttributeTok{filename =} \FunctionTok{file.path}\NormalTok{(out\_fig\_dir, }\StringTok{"Fig\_2a\_beta\_regression\_panel.png"}\NormalTok{),}
  \AttributeTok{plot =}\NormalTok{ panel\_beta,}
  \AttributeTok{width =} \DecValTok{12}\NormalTok{, }\AttributeTok{height =} \FloatTok{5.5}\NormalTok{, }\AttributeTok{dpi =} \DecValTok{800}
\NormalTok{)}


\DocumentationTok{\#\# {-}{-}{-}{-} 5) Diagnostic plots for Helotiales + Sebacinales {-}{-}{-}{-}}

\NormalTok{plot\_betareg\_diagnostics }\OtherTok{\textless{}{-}} \ControlFlowTok{function}\NormalTok{(model, data, }\AttributeTok{label\_prefix =} \StringTok{"Helotiales"}\NormalTok{) \{}

  \CommentTok{\# Ensure alignment (betareg expects the same row order it was fitted with)}
\NormalTok{  data }\OtherTok{\textless{}{-}}\NormalTok{ data[}\FunctionTok{complete.cases}\NormalTok{(data[, }\FunctionTok{c}\NormalTok{(}\StringTok{"elevation"}\NormalTok{,}\StringTok{"site"}\NormalTok{,}\StringTok{"Proportion"}\NormalTok{)]), , drop }\OtherTok{=} \ConstantTok{FALSE}\NormalTok{]}

  \CommentTok{\# residuals \& fitted}
\NormalTok{  res\_pear  }\OtherTok{\textless{}{-}} \FunctionTok{residuals}\NormalTok{(model, }\AttributeTok{type =} \StringTok{"pearson"}\NormalTok{)}
\NormalTok{  res\_quant }\OtherTok{\textless{}{-}} \FunctionTok{residuals}\NormalTok{(model, }\AttributeTok{type =} \StringTok{"quantile"}\NormalTok{)   }\CommentTok{\# approx N(0,1)}
\NormalTok{  fit       }\OtherTok{\textless{}{-}} \FunctionTok{fitted}\NormalTok{(model)}
\NormalTok{  hatv      }\OtherTok{\textless{}{-}} \FunctionTok{hatvalues}\NormalTok{(model)}

\NormalTok{  diagdf }\OtherTok{\textless{}{-}} \FunctionTok{data.frame}\NormalTok{(}
    \AttributeTok{fitted    =}\NormalTok{ fit,}
    \AttributeTok{res\_pear  =}\NormalTok{ res\_pear,}
    \AttributeTok{res\_quant =}\NormalTok{ res\_quant,}
    \AttributeTok{elevation =}\NormalTok{ data}\SpecialCharTok{$}\NormalTok{elevation,}
    \AttributeTok{site      =} \FunctionTok{as.factor}\NormalTok{(data}\SpecialCharTok{$}\NormalTok{site),}
    \AttributeTok{hat       =}\NormalTok{ hatv,}
    \AttributeTok{idx       =} \FunctionTok{seq\_len}\NormalTok{(}\FunctionTok{nrow}\NormalTok{(data))}
\NormalTok{  )}

  \CommentTok{\# (1) Residuals vs Fitted}
\NormalTok{  p1 }\OtherTok{\textless{}{-}} \FunctionTok{ggplot}\NormalTok{(diagdf, }\FunctionTok{aes}\NormalTok{(fitted, res\_pear)) }\SpecialCharTok{+}
    \FunctionTok{geom\_hline}\NormalTok{(}\AttributeTok{yintercept =} \DecValTok{0}\NormalTok{, }\AttributeTok{linetype =} \StringTok{"dashed"}\NormalTok{, }\AttributeTok{color =} \StringTok{"grey40"}\NormalTok{) }\SpecialCharTok{+}
    \FunctionTok{geom\_point}\NormalTok{(}\AttributeTok{alpha =} \FloatTok{0.7}\NormalTok{, }\AttributeTok{size =} \FloatTok{1.8}\NormalTok{) }\SpecialCharTok{+}
    \FunctionTok{geom\_smooth}\NormalTok{(}\AttributeTok{method =} \StringTok{"loess"}\NormalTok{, }\AttributeTok{se =} \ConstantTok{FALSE}\NormalTok{, }\AttributeTok{linewidth =} \FloatTok{0.5}\NormalTok{) }\SpecialCharTok{+}
    \FunctionTok{labs}\NormalTok{(}\AttributeTok{x =} \StringTok{"Fitted values"}\NormalTok{, }\AttributeTok{y =} \StringTok{"Pearson residuals"}\NormalTok{,}
         \AttributeTok{title =} \FunctionTok{paste}\NormalTok{(label\_prefix, }\StringTok{"– Residuals vs Fitted"}\NormalTok{)) }\SpecialCharTok{+}
    \FunctionTok{theme\_bw}\NormalTok{(}\AttributeTok{base\_size =} \DecValTok{9}\NormalTok{)}

  \CommentTok{\# (2) Residuals vs Elevation}
\NormalTok{  p2 }\OtherTok{\textless{}{-}} \FunctionTok{ggplot}\NormalTok{(diagdf, }\FunctionTok{aes}\NormalTok{(elevation, res\_pear)) }\SpecialCharTok{+}
    \FunctionTok{geom\_hline}\NormalTok{(}\AttributeTok{yintercept =} \DecValTok{0}\NormalTok{, }\AttributeTok{linetype =} \StringTok{"dashed"}\NormalTok{, }\AttributeTok{color =} \StringTok{"grey40"}\NormalTok{) }\SpecialCharTok{+}
    \FunctionTok{geom\_point}\NormalTok{(}\AttributeTok{alpha =} \FloatTok{0.7}\NormalTok{, }\AttributeTok{size =} \FloatTok{1.8}\NormalTok{) }\SpecialCharTok{+}
    \FunctionTok{geom\_smooth}\NormalTok{(}\AttributeTok{method =} \StringTok{"loess"}\NormalTok{, }\AttributeTok{se =} \ConstantTok{FALSE}\NormalTok{, }\AttributeTok{linewidth =} \FloatTok{0.5}\NormalTok{) }\SpecialCharTok{+}
    \FunctionTok{labs}\NormalTok{(}\AttributeTok{x =} \StringTok{"Elevation (m)"}\NormalTok{, }\AttributeTok{y =} \StringTok{"Pearson residuals"}\NormalTok{,}
         \AttributeTok{title =} \FunctionTok{paste}\NormalTok{(label\_prefix, }\StringTok{"– Residuals vs Elevation"}\NormalTok{)) }\SpecialCharTok{+}
    \FunctionTok{theme\_bw}\NormalTok{(}\AttributeTok{base\_size =} \DecValTok{9}\NormalTok{)}

  \CommentTok{\# (3) Q–Q plot of quantile residuals}
\NormalTok{  p3 }\OtherTok{\textless{}{-}} \FunctionTok{ggplot}\NormalTok{(diagdf, }\FunctionTok{aes}\NormalTok{(}\AttributeTok{sample =}\NormalTok{ res\_quant)) }\SpecialCharTok{+}
    \FunctionTok{stat\_qq}\NormalTok{(}\AttributeTok{size =} \FloatTok{1.4}\NormalTok{, }\AttributeTok{alpha =} \FloatTok{0.8}\NormalTok{) }\SpecialCharTok{+}
    \FunctionTok{stat\_qq\_line}\NormalTok{() }\SpecialCharTok{+}
    \FunctionTok{labs}\NormalTok{(}\AttributeTok{title =} \FunctionTok{paste}\NormalTok{(label\_prefix, }\StringTok{"– Q–Q Plot (quantile residuals)"}\NormalTok{),}
         \AttributeTok{x =} \StringTok{"Theoretical quantiles"}\NormalTok{, }\AttributeTok{y =} \StringTok{"Sample quantiles"}\NormalTok{) }\SpecialCharTok{+}
    \FunctionTok{theme\_bw}\NormalTok{(}\AttributeTok{base\_size =} \DecValTok{9}\NormalTok{)}

  \CommentTok{\# (4) Leverage}
\NormalTok{  hat\_thr }\OtherTok{\textless{}{-}} \DecValTok{2} \SpecialCharTok{*} \FunctionTok{mean}\NormalTok{(diagdf}\SpecialCharTok{$}\NormalTok{hat, }\AttributeTok{na.rm =} \ConstantTok{TRUE}\NormalTok{)}
\NormalTok{  p4 }\OtherTok{\textless{}{-}} \FunctionTok{ggplot}\NormalTok{(diagdf, }\FunctionTok{aes}\NormalTok{(idx, hat)) }\SpecialCharTok{+}
    \FunctionTok{geom\_point}\NormalTok{(}\AttributeTok{alpha =} \FloatTok{0.85}\NormalTok{, }\AttributeTok{size =} \FloatTok{1.8}\NormalTok{) }\SpecialCharTok{+}
    \FunctionTok{geom\_hline}\NormalTok{(}\AttributeTok{yintercept =}\NormalTok{ hat\_thr, }\AttributeTok{color =} \StringTok{"red"}\NormalTok{, }\AttributeTok{linetype =} \StringTok{"dashed"}\NormalTok{) }\SpecialCharTok{+}
    \FunctionTok{labs}\NormalTok{(}\AttributeTok{x =} \StringTok{"Observation index"}\NormalTok{, }\AttributeTok{y =} \StringTok{"Leverage (hat values)"}\NormalTok{,}
         \AttributeTok{title =} \FunctionTok{paste}\NormalTok{(label\_prefix, }\StringTok{"– Leverage"}\NormalTok{)) }\SpecialCharTok{+}
    \FunctionTok{theme\_bw}\NormalTok{(}\AttributeTok{base\_size =} \DecValTok{9}\NormalTok{)}

\NormalTok{  (p1 }\SpecialCharTok{|}\NormalTok{ p2) }\SpecialCharTok{/}\NormalTok{ (p3 }\SpecialCharTok{|}\NormalTok{ p4) }\SpecialCharTok{+}
\NormalTok{    patchwork}\SpecialCharTok{::}\FunctionTok{plot\_annotation}\NormalTok{(}\AttributeTok{title =} \FunctionTok{paste}\NormalTok{(label\_prefix, }\StringTok{"model diagnostics"}\NormalTok{)) }\SpecialCharTok{\&}
    \FunctionTok{theme}\NormalTok{(}\AttributeTok{plot.title =} \FunctionTok{element\_text}\NormalTok{(}\AttributeTok{face =} \StringTok{"bold"}\NormalTok{, }\AttributeTok{size =} \DecValTok{10}\NormalTok{, }\AttributeTok{hjust =} \DecValTok{0}\NormalTok{))}
\NormalTok{\}}

\CommentTok{\# Use your existing objects from Steps 1–3}
\NormalTok{diag\_helot }\OtherTok{\textless{}{-}} \FunctionTok{plot\_betareg\_diagnostics}\NormalTok{(mod\_list[[}\StringTok{"helotiales"}\NormalTok{]],  df\_list[[}\StringTok{"helotiales"}\NormalTok{]],  }\StringTok{"Helotiales"}\NormalTok{)}
\NormalTok{diag\_seba  }\OtherTok{\textless{}{-}} \FunctionTok{plot\_betareg\_diagnostics}\NormalTok{(mod\_list[[}\StringTok{"sebacinales"}\NormalTok{]], df\_list[[}\StringTok{"sebacinales"}\NormalTok{]], }\StringTok{"Sebacinales"}\NormalTok{)}

\FunctionTok{ggsave}\NormalTok{(}
  \AttributeTok{filename =} \FunctionTok{file.path}\NormalTok{(out\_fig\_dir, }\StringTok{"Fig\_S6\_betareg\_diagnostics\_helotiales.png"}\NormalTok{),}
  \AttributeTok{plot =}\NormalTok{ diag\_helot, }\AttributeTok{width =} \FloatTok{8.5}\NormalTok{, }\AttributeTok{height =} \FloatTok{6.5}\NormalTok{, }\AttributeTok{dpi =} \DecValTok{600}
\NormalTok{)}

\FunctionTok{ggsave}\NormalTok{(}
  \AttributeTok{filename =} \FunctionTok{file.path}\NormalTok{(out\_fig\_dir, }\StringTok{"Fig\_S7\_betareg\_diagnostics\_sebacinales.png"}\NormalTok{),}
  \AttributeTok{plot =}\NormalTok{ diag\_seba, }\AttributeTok{width =} \FloatTok{8.5}\NormalTok{, }\AttributeTok{height =} \FloatTok{6.5}\NormalTok{, }\AttributeTok{dpi =} \DecValTok{600}
\NormalTok{)}
\end{Highlighting}
\end{Shaded}

\subsection{Functional annotation with
FUNGuild}\label{functional-annotation-with-funguild}

Functional guilds from root samples were assigned using FUNGuild based
on the curated taxonomic strings exported from the phyloseq object. OTUs
with a valid confidence ranking in the FUNGuild output were retained for
downstream summaries. Guild labels were additionally collapsed into a
primary-guild hierarchy (Ericoid mycorrhizal \textgreater{} other
mycorrhizal \textgreater{} endophyte \textgreater{} pathotroph
\textgreater{} saprotroph \textgreater{} other) to simplify
interpretation. Figure S8 summarizes mean guild-associated read
abundance per sample across habitats.

\includegraphics{figures/Fig_S9_FunGuild.png}

\textbf{Figure S8}: Mean relative read abundance per sample of fungal
functional guilds across forest, subpáramo, and páramo habitats, based
on FUNGuild annotations. Only root samples. \emph{Code: see ``FUNGuild
functional annotation'' below.}

\begin{Shaded}
\begin{Highlighting}[]
\CommentTok{\# Functional guild assignment was performed using FUNGuild prior to figure generation.}
\CommentTok{\#}
\CommentTok{\# The full annotation workflow is available in the GitHub repository:}
\CommentTok{\#   scripts/funguild\_script.R}
\CommentTok{\#}
\CommentTok{\# Input:}
\CommentTok{\#  {-} Taxonomically annotated ASV/OTU table}
\CommentTok{\#}
\CommentTok{\# Output:}
\CommentTok{\#  {-} Table of guild assignments per taxon}
\CommentTok{\#  {-} Mean read abundances per sample and habitat}
\CommentTok{\#}
\CommentTok{\# Figure S8 was generated from these outputs and is included above as a PNG.}
\end{Highlighting}
\end{Shaded}

\subsection{Generalized Linear Latent Variable Model
(GLLVM)}\label{generalized-linear-latent-variable-model-gllvm}

We modelled OTU counts using a generalized linear latent variable model
(GLLVM; negative binomial distribution, log link), with habitat (forest,
subpáramo, páramo) as a fixed effect, log library size as an offset, and
random intercepts for site and individual plants (Unique\_ID). One
latent variable was included to account for residual correlation among
taxa. The final model (log-likelihood = −37,616; AIC = 80,295) provided
the best balance between fit and parsimony. It substantially
outperformed earlier versions with nested random effects (AIC =
186,832), and attempts to increase the number of latent variables to two
worsened performance (ΔAIC ≈ 12,700) and produced a singular information
matrix, indicating over-parameterization. Simplifying the random-effects
structure and filtering rare taxa also improved convergence and reduced
overdispersion while retaining ecological signal.

Model fit was evaluated using Dunn--Smyth residual diagnostics (Fig.
S9). Habitat coefficients (β) were extracted for each OTU from the
fixed-effects matrix. To obtain genus-level responses, OTUs were
restricted to those assigned to a genus (excluding incertae sedis) and
weighted by their total read abundance across samples. For each genus,
we computed the abundance-weighted mean β for páramo and subpáramo
relative to forest. Uncertainty was summarized using a weighted
bootstrap (R = 500) resampling OTUs within genera with probabilities
proportional to total abundance; 2.5--97.5\% quantiles were used as 95\%
confidence intervals. These summaries were used to generate the heatmap
and barplot shown in Figure 4. The full OTU-level coefficient table is
provided in Data\_S2\_GLLVM\_table.csv.

\includegraphics{figures/Fig_S10_GLLVM_diagnostics.png}

\textbf{Figure S9}: Diagnostic plots for the negative-binomial GLLVM
(Dunn--Smyth residuals). \emph{Code: see ``GLLVM model fitting +
coefficient summaries'' below.}

For transparency, we provide a short preview of the genus-level GLLVM
summaries below. The complete OTU- and genus-level coefficient table is
available as Data S2 (Data\_S2\_GLLVM\_table.csv).

\begin{longtable}[]{@{}
  >{\raggedright\arraybackslash}p{(\columnwidth - 12\tabcolsep) * \real{0.1200}}
  >{\raggedright\arraybackslash}p{(\columnwidth - 12\tabcolsep) * \real{0.1040}}
  >{\raggedleft\arraybackslash}p{(\columnwidth - 12\tabcolsep) * \real{0.0560}}
  >{\raggedleft\arraybackslash}p{(\columnwidth - 12\tabcolsep) * \real{0.1520}}
  >{\raggedleft\arraybackslash}p{(\columnwidth - 12\tabcolsep) * \real{0.1760}}
  >{\raggedleft\arraybackslash}p{(\columnwidth - 12\tabcolsep) * \real{0.1840}}
  >{\raggedleft\arraybackslash}p{(\columnwidth - 12\tabcolsep) * \real{0.2080}}@{}}
\caption{Data S2 (preview). Selected columns from the GLLVM genus-level
summary. The full table with all metrics is provided as a CSV
file.}\tabularnewline
\toprule\noalign{}
\begin{minipage}[b]{\linewidth}\raggedright
Order
\end{minipage} & \begin{minipage}[b]{\linewidth}\raggedright
Genus
\end{minipage} & \begin{minipage}[b]{\linewidth}\raggedleft
n\_OTUs
\end{minipage} & \begin{minipage}[b]{\linewidth}\raggedleft
w\_mean\_beta\_paramo
\end{minipage} & \begin{minipage}[b]{\linewidth}\raggedleft
w\_mean\_beta\_subparamo
\end{minipage} & \begin{minipage}[b]{\linewidth}\raggedleft
perc\_enriched\_paramo\_w
\end{minipage} & \begin{minipage}[b]{\linewidth}\raggedleft
perc\_enriched\_subparamo\_w
\end{minipage} \\
\midrule\noalign{}
\endfirsthead
\toprule\noalign{}
\begin{minipage}[b]{\linewidth}\raggedright
Order
\end{minipage} & \begin{minipage}[b]{\linewidth}\raggedright
Genus
\end{minipage} & \begin{minipage}[b]{\linewidth}\raggedleft
n\_OTUs
\end{minipage} & \begin{minipage}[b]{\linewidth}\raggedleft
w\_mean\_beta\_paramo
\end{minipage} & \begin{minipage}[b]{\linewidth}\raggedleft
w\_mean\_beta\_subparamo
\end{minipage} & \begin{minipage}[b]{\linewidth}\raggedleft
perc\_enriched\_paramo\_w
\end{minipage} & \begin{minipage}[b]{\linewidth}\raggedleft
perc\_enriched\_subparamo\_w
\end{minipage} \\
\midrule\noalign{}
\endhead
\bottomrule\noalign{}
\endlastfoot
Coniochaetales & Coniochaeta & 2 & 254.379 & -25.536 & 2.048010e+115 &
-1.000000e+02 \\
Helotiales & Acephala & 1 & 78.144 & 69.862 & 8.663205e+35 &
2.190394e+32 \\
Helotiales & Hyaloscypha & 3 & 26.861 & 29.053 & 3.743955e+14 &
6.764441e+13 \\
Helotiales & Pezicula & 5 & 26.640 & 29.280 & 1.824961e+09 &
-9.894700e+01 \\
Helotiales & Lachnum & 3 & -1.529 & 9.013 & 1.016873e+08 &
2.107609e+04 \\
Sclerococcales & Sclerococcum & 4 & 24.669 & -8.648 & 2.210442e+06 &
-9.989400e+01 \\
Mortierellales & Mortierella & 5 & -7.650 & -34.063 & 1.011728e+04 &
-1.000000e+02 \\
Helotiales & Gyoerffyella & 3 & 2.739 & 0.188 & 1.021700e+01 &
-9.967900e+01 \\
Ostropales & Cryptodiscus & 3 & -15.238 & 12.611 & -1.947200e+01 &
6.019926e+04 \\
Helotiales & Pezoloma & 2 & -6.125 & -4.096 & -9.964100e+01 &
-9.684900e+01 \\
\end{longtable}

\begin{Shaded}
\begin{Highlighting}[]
\CommentTok{\# GLLVM analysis was executed on the server due to computational cost.}
\CommentTok{\# The complete workflow (data filtering, model fitting, diagnostics, and genus{-}level summaries) is provided in the repository:}
\CommentTok{\#   scripts/gllvm\_analysis.R}
\CommentTok{\#}
\CommentTok{\# Key outputs saved:}
\CommentTok{\#   {-} figures/Fig\_S9\_GLLVM\_diagnostics.png}
\CommentTok{\#   {-} tables/File\_S2\_GLLVM\_table.csv}
\CommentTok{\#   {-} objects/fit\_nb\_2.rds  \# fitted model object}
\end{Highlighting}
\end{Shaded}

\subsection{Diversity analyses with
iNEXT3D}\label{diversity-analyses-with-inext3d}

We quantified habitat-associated patterns in taxonomic diversity (TD)
and phylogenetic diversity (PD) using iNEXT3D and Hill numbers (q = 0,
1, 2). Diversity estimates were computed from incidence
(presence/absence) matrices at the habitat level and evaluated under
rarefaction/extrapolation with bootstrap confidence intervals (nboot =
500). Phylogenetic diversity was estimated as meanPD using a pruned
phylogeny matched to the observed taxa. Diversity curves are presented
in the main text (Fig. 4), while asymptotic summaries are provided here
(Tables S11--S12).

\textbf{Table S11}: Taxonomic diversity summary

\begin{longtable}[]{@{}
  >{\raggedright\arraybackslash}p{(\columnwidth - 12\tabcolsep) * \real{0.1507}}
  >{\raggedright\arraybackslash}p{(\columnwidth - 12\tabcolsep) * \real{0.2466}}
  >{\raggedleft\arraybackslash}p{(\columnwidth - 12\tabcolsep) * \real{0.1233}}
  >{\raggedleft\arraybackslash}p{(\columnwidth - 12\tabcolsep) * \real{0.1233}}
  >{\raggedleft\arraybackslash}p{(\columnwidth - 12\tabcolsep) * \real{0.1096}}
  >{\raggedleft\arraybackslash}p{(\columnwidth - 12\tabcolsep) * \real{0.1233}}
  >{\raggedleft\arraybackslash}p{(\columnwidth - 12\tabcolsep) * \real{0.1233}}@{}}
\caption{Table S11. Taxonomic diversity (TD) summary from iNEXT3D across
habitats (q = 0, 1, 2; nboot = 500). Observed and asymptotic estimates
are reported with standard errors, 95\% confidence intervals, and sample
coverage at observed effort and at double effort.}\tabularnewline
\toprule\noalign{}
\begin{minipage}[b]{\linewidth}\raggedright
Assemblage
\end{minipage} & \begin{minipage}[b]{\linewidth}\raggedright
qTD
\end{minipage} & \begin{minipage}[b]{\linewidth}\raggedleft
TD\_obs
\end{minipage} & \begin{minipage}[b]{\linewidth}\raggedleft
TD\_asy
\end{minipage} & \begin{minipage}[b]{\linewidth}\raggedleft
s.e.
\end{minipage} & \begin{minipage}[b]{\linewidth}\raggedleft
qTD.LCL
\end{minipage} & \begin{minipage}[b]{\linewidth}\raggedleft
qTD.UCL
\end{minipage} \\
\midrule\noalign{}
\endfirsthead
\toprule\noalign{}
\begin{minipage}[b]{\linewidth}\raggedright
Assemblage
\end{minipage} & \begin{minipage}[b]{\linewidth}\raggedright
qTD
\end{minipage} & \begin{minipage}[b]{\linewidth}\raggedleft
TD\_obs
\end{minipage} & \begin{minipage}[b]{\linewidth}\raggedleft
TD\_asy
\end{minipage} & \begin{minipage}[b]{\linewidth}\raggedleft
s.e.
\end{minipage} & \begin{minipage}[b]{\linewidth}\raggedleft
qTD.LCL
\end{minipage} & \begin{minipage}[b]{\linewidth}\raggedleft
qTD.UCL
\end{minipage} \\
\midrule\noalign{}
\endhead
\bottomrule\noalign{}
\endlastfoot
Forest & Species richness & 2521.000 & 4581.983 & 125.848 & 4335.326 &
4828.640 \\
Forest & Shannon diversity & 1810.823 & 2927.463 & 51.242 & 2827.030 &
3027.896 \\
Forest & Simpson diversity & 1158.528 & 1467.726 & 34.168 & 1400.758 &
1534.695 \\
Páramo & Species richness & 901.000 & 1945.702 & 95.214 & 1759.086 &
2132.317 \\
Páramo & Shannon diversity & 693.768 & 1316.325 & 43.456 & 1231.153 &
1401.497 \\
Páramo & Simpson diversity & 474.571 & 661.005 & 30.584 & 601.061 &
720.948 \\
Subpáramo & Species richness & 1533.000 & 3059.820 & 108.142 & 2847.866
& 3271.775 \\
Subpáramo & Shannon diversity & 1225.295 & 2341.860 & 58.351 & 2227.494
& 2456.226 \\
Subpáramo & Simpson diversity & 846.718 & 1247.075 & 49.323 & 1150.403 &
1343.747 \\
\end{longtable}

\textbf{Table S12}: Phylogenetic diversity summary

\begin{Shaded}
\begin{Highlighting}[]
\CommentTok{\#| echo: false}
\FunctionTok{library}\NormalTok{(knitr)}

\NormalTok{tab\_s12 }\OtherTok{\textless{}{-}} \FunctionTok{read.csv}\NormalTok{(}\StringTok{"tables/Table\_S12\_iNEXT3D\_phylogenetic\_diversity.csv"}\NormalTok{)}

\FunctionTok{kable}\NormalTok{(}
\NormalTok{  tab\_s12,}
  \AttributeTok{digits =} \DecValTok{3}\NormalTok{,}
  \AttributeTok{caption =} \StringTok{"Table S12. Phylogenetic diversity (PD; meanPD) summary from iNEXT3D across habitats (q = 0, 1, 2; nboot = 500), including coverage and effective lineage estimates under rarefaction/extrapolation."}
\NormalTok{)}
\end{Highlighting}
\end{Shaded}

\begin{longtable}[]{@{}
  >{\raggedright\arraybackslash}p{(\columnwidth - 16\tabcolsep) * \real{0.1389}}
  >{\raggedright\arraybackslash}p{(\columnwidth - 16\tabcolsep) * \real{0.1250}}
  >{\raggedleft\arraybackslash}p{(\columnwidth - 16\tabcolsep) * \real{0.1111}}
  >{\raggedleft\arraybackslash}p{(\columnwidth - 16\tabcolsep) * \real{0.1111}}
  >{\raggedleft\arraybackslash}p{(\columnwidth - 16\tabcolsep) * \real{0.0833}}
  >{\raggedleft\arraybackslash}p{(\columnwidth - 16\tabcolsep) * \real{0.1111}}
  >{\raggedleft\arraybackslash}p{(\columnwidth - 16\tabcolsep) * \real{0.1111}}
  >{\raggedleft\arraybackslash}p{(\columnwidth - 16\tabcolsep) * \real{0.1111}}
  >{\raggedright\arraybackslash}p{(\columnwidth - 16\tabcolsep) * \real{0.0972}}@{}}
\caption{Table S12. Phylogenetic diversity (PD; meanPD) summary from
iNEXT3D across habitats (q = 0, 1, 2; nboot = 500), including coverage
and effective lineage estimates under
rarefaction/extrapolation.}\tabularnewline
\toprule\noalign{}
\begin{minipage}[b]{\linewidth}\raggedright
Habitat
\end{minipage} & \begin{minipage}[b]{\linewidth}\raggedright
qPD
\end{minipage} & \begin{minipage}[b]{\linewidth}\raggedleft
PD\_obs
\end{minipage} & \begin{minipage}[b]{\linewidth}\raggedleft
PD\_asy
\end{minipage} & \begin{minipage}[b]{\linewidth}\raggedleft
s.e.
\end{minipage} & \begin{minipage}[b]{\linewidth}\raggedleft
qPD.LCL
\end{minipage} & \begin{minipage}[b]{\linewidth}\raggedleft
qPD.UCL
\end{minipage} & \begin{minipage}[b]{\linewidth}\raggedleft
Reftime
\end{minipage} & \begin{minipage}[b]{\linewidth}\raggedright
Type
\end{minipage} \\
\midrule\noalign{}
\endfirsthead
\toprule\noalign{}
\begin{minipage}[b]{\linewidth}\raggedright
Habitat
\end{minipage} & \begin{minipage}[b]{\linewidth}\raggedright
qPD
\end{minipage} & \begin{minipage}[b]{\linewidth}\raggedleft
PD\_obs
\end{minipage} & \begin{minipage}[b]{\linewidth}\raggedleft
PD\_asy
\end{minipage} & \begin{minipage}[b]{\linewidth}\raggedleft
s.e.
\end{minipage} & \begin{minipage}[b]{\linewidth}\raggedleft
qPD.LCL
\end{minipage} & \begin{minipage}[b]{\linewidth}\raggedleft
qPD.UCL
\end{minipage} & \begin{minipage}[b]{\linewidth}\raggedleft
Reftime
\end{minipage} & \begin{minipage}[b]{\linewidth}\raggedright
Type
\end{minipage} \\
\midrule\noalign{}
\endhead
\bottomrule\noalign{}
\endlastfoot
Forest & q = 0 PD & 185.553 & 283.343 & 7.643 & 268.362 & 298.324 &
0.873 & meanPD \\
Forest & q = 1 PD & 96.045 & 115.843 & 1.427 & 113.045 & 118.641 & 0.873
& meanPD \\
Forest & q = 2 PD & 53.644 & 56.697 & 0.585 & 55.551 & 57.842 & 0.873 &
meanPD \\
Páramo & q = 0 PD & 81.971 & 134.076 & 6.501 & 121.334 & 146.818 & 0.873
& meanPD \\
Páramo & q = 1 PD & 46.984 & 60.900 & 1.321 & 58.310 & 63.489 & 0.873 &
meanPD \\
Páramo & q = 2 PD & 27.101 & 29.356 & 0.529 & 28.320 & 30.392 & 0.873 &
meanPD \\
Subpáramo & q = 0 PD & 132.936 & 213.016 & 7.909 & 197.514 & 228.518 &
0.873 & meanPD \\
Subpáramo & q = 1 PD & 75.157 & 98.360 & 1.843 & 94.748 & 101.973 &
0.873 & meanPD \\
Subpáramo & q = 2 PD & 39.842 & 43.233 & 0.715 & 41.831 & 44.635 & 0.873
& meanPD \\
\end{longtable}

\begin{Shaded}
\begin{Highlighting}[]
\FunctionTok{library}\NormalTok{(phyloseq)}
\FunctionTok{library}\NormalTok{(dplyr)}
\FunctionTok{library}\NormalTok{(tidyr)}
\FunctionTok{library}\NormalTok{(ggplot2)}
\FunctionTok{library}\NormalTok{(iNEXT}\FloatTok{.3}\NormalTok{D)  }
\FunctionTok{library}\NormalTok{(ape)        }
\FunctionTok{library}\NormalTok{(knitr)}
\FunctionTok{library}\NormalTok{(phangorn)}


\FunctionTok{set.seed}\NormalTok{(}\DecValTok{1}\NormalTok{)}

\DocumentationTok{\#\# {-}{-}{-}{-} shared habitat labeling {-}{-}{-}{-}}
\NormalTok{labs\_map  }\OtherTok{\textless{}{-}} \FunctionTok{c}\NormalTok{(}\AttributeTok{forest =} \StringTok{"Forest"}\NormalTok{, }\AttributeTok{subparamo =} \StringTok{"Subpáramo"}\NormalTok{, }\AttributeTok{paramo =} \StringTok{"Páramo"}\NormalTok{)}
\NormalTok{lvl\_order }\OtherTok{\textless{}{-}} \FunctionTok{c}\NormalTok{(}\StringTok{"Forest"}\NormalTok{, }\StringTok{"Subpáramo"}\NormalTok{, }\StringTok{"Páramo"}\NormalTok{)}

\NormalTok{relabel\_inext }\OtherTok{\textless{}{-}} \ControlFlowTok{function}\NormalTok{(x) \{}
  \CommentTok{\# iNEXT3D objects store multiple data.frames with "Assemblage"}
\NormalTok{  f }\OtherTok{\textless{}{-}} \ControlFlowTok{function}\NormalTok{(df)\{}
    \ControlFlowTok{if}\NormalTok{ (}\SpecialCharTok{!}\NormalTok{(}\StringTok{"Assemblage"} \SpecialCharTok{\%in\%} \FunctionTok{names}\NormalTok{(df))) }\FunctionTok{return}\NormalTok{(df)}
\NormalTok{    df}\SpecialCharTok{$}\NormalTok{Assemblage }\OtherTok{\textless{}{-}} \FunctionTok{as.character}\NormalTok{(df}\SpecialCharTok{$}\NormalTok{Assemblage)}
\NormalTok{    df}\SpecialCharTok{$}\NormalTok{Assemblage }\OtherTok{\textless{}{-}}\NormalTok{ labs\_map[df}\SpecialCharTok{$}\NormalTok{Assemblage]}
\NormalTok{    df}\SpecialCharTok{$}\NormalTok{Assemblage }\OtherTok{\textless{}{-}} \FunctionTok{factor}\NormalTok{(df}\SpecialCharTok{$}\NormalTok{Assemblage, }\AttributeTok{levels =}\NormalTok{ lvl\_order)}
\NormalTok{    df}
\NormalTok{  \}}
\NormalTok{  x}\SpecialCharTok{$}\NormalTok{TDInfo   }\OtherTok{\textless{}{-}} \FunctionTok{f}\NormalTok{(}\FunctionTok{as.data.frame}\NormalTok{(x}\SpecialCharTok{$}\NormalTok{TDInfo))}
\NormalTok{  x}\SpecialCharTok{$}\NormalTok{TDAsyEst }\OtherTok{\textless{}{-}} \FunctionTok{f}\NormalTok{(}\FunctionTok{as.data.frame}\NormalTok{(x}\SpecialCharTok{$}\NormalTok{TDAsyEst))}
  \ControlFlowTok{if}\NormalTok{ (}\SpecialCharTok{!}\FunctionTok{is.null}\NormalTok{(x}\SpecialCharTok{$}\NormalTok{TDiNextEst)) \{}
    \ControlFlowTok{for}\NormalTok{ (nm }\ControlFlowTok{in} \FunctionTok{names}\NormalTok{(x}\SpecialCharTok{$}\NormalTok{TDiNextEst)) x}\SpecialCharTok{$}\NormalTok{TDiNextEst[[nm]] }\OtherTok{\textless{}{-}} \FunctionTok{f}\NormalTok{(}\FunctionTok{as.data.frame}\NormalTok{(x}\SpecialCharTok{$}\NormalTok{TDiNextEst[[nm]]))}
\NormalTok{  \}}
\NormalTok{  x}
\NormalTok{\}}


\DocumentationTok{\#\# =================================================}
\DocumentationTok{\#\#  A) TAXONOMIC DIVERSITY (TD)  {-}{-}{-}{-} Table S10 + TD plot}
\DocumentationTok{\#\# =================================================}

\NormalTok{ps\_td }\OtherTok{\textless{}{-}}\NormalTok{ tree\_ps}

\FunctionTok{stopifnot}\NormalTok{(}\FunctionTok{inherits}\NormalTok{(ps\_td, }\StringTok{"phyloseq"}\NormalTok{))}

\NormalTok{sd }\OtherTok{\textless{}{-}} \FunctionTok{as.data.frame}\NormalTok{(}\FunctionTok{sample\_data}\NormalTok{(ps\_td))}
\FunctionTok{stopifnot}\NormalTok{(}\StringTok{"habitat"} \SpecialCharTok{\%in\%} \FunctionTok{names}\NormalTok{(sd))}

\CommentTok{\# build taxa x samples incidence matrix}
\NormalTok{OTU }\OtherTok{\textless{}{-}} \FunctionTok{as}\NormalTok{(}\FunctionTok{otu\_table}\NormalTok{(ps\_td), }\StringTok{"matrix"}\NormalTok{)}
\ControlFlowTok{if}\NormalTok{ (}\SpecialCharTok{!}\FunctionTok{taxa\_are\_rows}\NormalTok{(ps\_td)) OTU }\OtherTok{\textless{}{-}} \FunctionTok{t}\NormalTok{(OTU)   }\CommentTok{\# taxa x samples}

\CommentTok{\# split samples by habitat}
\NormalTok{inc\_by\_hab\_td }\OtherTok{\textless{}{-}} \FunctionTok{lapply}\NormalTok{(}\FunctionTok{split}\NormalTok{(}\FunctionTok{rownames}\NormalTok{(sd), sd}\SpecialCharTok{$}\NormalTok{habitat), }\ControlFlowTok{function}\NormalTok{(samps)\{}
\NormalTok{  M }\OtherTok{\textless{}{-}}\NormalTok{ OTU[, }\FunctionTok{colnames}\NormalTok{(OTU) }\SpecialCharTok{\%in\%}\NormalTok{ samps, drop }\OtherTok{=} \ConstantTok{FALSE}\NormalTok{]}
\NormalTok{  M[M }\SpecialCharTok{\textgreater{}} \DecValTok{0}\NormalTok{] }\OtherTok{\textless{}{-}} \DecValTok{1}
  \FunctionTok{storage.mode}\NormalTok{(M) }\OtherTok{\textless{}{-}} \StringTok{"numeric"}
\NormalTok{  M}
\NormalTok{\})}

\CommentTok{\# keep only expected habitats and non{-}empty}
\NormalTok{inc\_by\_hab\_td }\OtherTok{\textless{}{-}}\NormalTok{ inc\_by\_hab\_td[}\FunctionTok{names}\NormalTok{(inc\_by\_hab\_td) }\SpecialCharTok{\%in\%} \FunctionTok{names}\NormalTok{(labs\_map)]}
\NormalTok{inc\_by\_hab\_td }\OtherTok{\textless{}{-}}\NormalTok{ inc\_by\_hab\_td[}\FunctionTok{vapply}\NormalTok{(inc\_by\_hab\_td, }\ControlFlowTok{function}\NormalTok{(M) }\FunctionTok{nrow}\NormalTok{(M) }\SpecialCharTok{\textgreater{}} \DecValTok{0} \SpecialCharTok{\&\&} \FunctionTok{ncol}\NormalTok{(M) }\SpecialCharTok{\textgreater{}} \DecValTok{0}\NormalTok{, }\ConstantTok{TRUE}\NormalTok{)]}
\FunctionTok{stopifnot}\NormalTok{(}\FunctionTok{length}\NormalTok{(inc\_by\_hab\_td) }\SpecialCharTok{\textgreater{}=} \DecValTok{2}\NormalTok{)}

\NormalTok{out\_TD }\OtherTok{\textless{}{-}} \FunctionTok{iNEXT3D}\NormalTok{(}
  \AttributeTok{data      =}\NormalTok{ inc\_by\_hab\_td,}
  \AttributeTok{diversity =} \StringTok{"TD"}\NormalTok{,}
  \AttributeTok{q         =} \FunctionTok{c}\NormalTok{(}\DecValTok{0}\NormalTok{,}\DecValTok{1}\NormalTok{,}\DecValTok{2}\NormalTok{),}
  \AttributeTok{datatype  =} \StringTok{"incidence\_raw"}\NormalTok{,}
  \AttributeTok{nboot     =} \DecValTok{500}
\NormalTok{)}

\CommentTok{\# relabel habitat names + ordering for plotting/tables}
\NormalTok{out\_TD }\OtherTok{\textless{}{-}} \FunctionTok{relabel\_inext}\NormalTok{(out\_TD)}

\CommentTok{\# {-}{-}{-}{-} Figure (TD curves) {-}{-}{-}{-}}
\NormalTok{p\_TD }\OtherTok{\textless{}{-}} \FunctionTok{ggiNEXT3D}\NormalTok{(out\_TD, }\AttributeTok{type =} \DecValTok{1}\NormalTok{, }\AttributeTok{facet.var =} \StringTok{"Order.q"}\NormalTok{) }\SpecialCharTok{+}
  \FunctionTok{labs}\NormalTok{(}\AttributeTok{x =} \StringTok{"Sampling units"}\NormalTok{, }\AttributeTok{y =} \StringTok{"Taxonomic diversity"}\NormalTok{) }\SpecialCharTok{+}
  \FunctionTok{theme\_minimal}\NormalTok{(}\AttributeTok{base\_size =} \DecValTok{14}\NormalTok{) }\SpecialCharTok{+}
  \FunctionTok{theme}\NormalTok{(}\AttributeTok{legend.title =} \FunctionTok{element\_blank}\NormalTok{())}

\FunctionTok{ggsave}\NormalTok{(}
  \AttributeTok{filename =} \FunctionTok{file.path}\NormalTok{(out\_fig\_dir, }\StringTok{"Fig\_4B\_iNEXT3D\_TD.png"}\NormalTok{),}
  \AttributeTok{plot =}\NormalTok{ p\_TD, }\AttributeTok{width =} \DecValTok{10}\NormalTok{, }\AttributeTok{height =} \DecValTok{6}\NormalTok{, }\AttributeTok{dpi =} \DecValTok{800}\NormalTok{, }\AttributeTok{bg =} \StringTok{"white"}
\NormalTok{)}

\CommentTok{\# {-}{-}{-}{-} Table S10 (TD summary)}
\NormalTok{tab\_s10 }\OtherTok{\textless{}{-}} \FunctionTok{as.data.frame}\NormalTok{(out\_TD}\SpecialCharTok{$}\NormalTok{TDAsyEst)}

\CommentTok{\# Optional: make “Diversity\_order” readable}
\NormalTok{tab\_s10 }\OtherTok{\textless{}{-}}\NormalTok{ tab\_s10 }\SpecialCharTok{\%\textgreater{}\%}
  \FunctionTok{mutate}\NormalTok{(}
    \AttributeTok{Diversity\_order =} \FunctionTok{case\_when}\NormalTok{(}
\NormalTok{      qTD }\SpecialCharTok{==} \DecValTok{0} \SpecialCharTok{\textasciitilde{}} \StringTok{"Species richness"}\NormalTok{,}
\NormalTok{      qTD }\SpecialCharTok{==} \DecValTok{1} \SpecialCharTok{\textasciitilde{}} \StringTok{"Shannon diversity"}\NormalTok{,}
\NormalTok{      qTD }\SpecialCharTok{==} \DecValTok{2} \SpecialCharTok{\textasciitilde{}} \StringTok{"Simpson diversity"}\NormalTok{,}
      \ConstantTok{TRUE} \SpecialCharTok{\textasciitilde{}} \FunctionTok{as.character}\NormalTok{(qTD)}
\NormalTok{    )}
\NormalTok{  ) }\SpecialCharTok{\%\textgreater{}\%}
  \FunctionTok{rename}\NormalTok{(}
    \AttributeTok{Habitat =}\NormalTok{ Assemblage}
\NormalTok{  )}

\FunctionTok{write.csv}\NormalTok{(}
\NormalTok{  tab\_s10,}
  \FunctionTok{file.path}\NormalTok{(out\_tab\_dir, }\StringTok{"Table\_S11\_iNEXT3D\_taxonomic\_diversity.csv"}\NormalTok{),}
  \AttributeTok{row.names =} \ConstantTok{FALSE}
\NormalTok{)}


\DocumentationTok{\#\# ============================================================}
\DocumentationTok{\#\#  B) PHYLOGENETIC DIVERSITY (PD) {-}{-}{-}{-} Table S11 + PD plot}
\DocumentationTok{\#\# ============================================================}

\NormalTok{ps\_pd }\OtherTok{\textless{}{-}}\NormalTok{ tree\_ps }\CommentTok{\#phyloseq object with phy tree}
\FunctionTok{stopifnot}\NormalTok{(}\FunctionTok{inherits}\NormalTok{(ps\_pd, }\StringTok{"phyloseq"}\NormalTok{))}
\FunctionTok{stopifnot}\NormalTok{(}\SpecialCharTok{!}\FunctionTok{is.null}\NormalTok{(}\FunctionTok{phy\_tree}\NormalTok{(ps\_pd, }\AttributeTok{errorIfNULL =} \ConstantTok{FALSE}\NormalTok{)))}

\NormalTok{sd2 }\OtherTok{\textless{}{-}} \FunctionTok{as.data.frame}\NormalTok{(}\FunctionTok{sample\_data}\NormalTok{(ps\_pd))}
\FunctionTok{stopifnot}\NormalTok{(}\StringTok{"habitat"} \SpecialCharTok{\%in\%} \FunctionTok{names}\NormalTok{(sd2))}

\CommentTok{\# Fix mislabel if present}
\NormalTok{sd2}\SpecialCharTok{$}\NormalTok{habitat }\OtherTok{\textless{}{-}} \FunctionTok{as.character}\NormalTok{(sd2}\SpecialCharTok{$}\NormalTok{habitat)}
\NormalTok{sd2}\SpecialCharTok{$}\NormalTok{habitat[sd2}\SpecialCharTok{$}\NormalTok{habitat }\SpecialCharTok{==} \StringTok{"pasture"}\NormalTok{] }\OtherTok{\textless{}{-}} \StringTok{"forest"}
\FunctionTok{sample\_data}\NormalTok{(ps\_pd)}\SpecialCharTok{$}\NormalTok{habitat }\OtherTok{\textless{}{-}} \FunctionTok{factor}\NormalTok{(sd2}\SpecialCharTok{$}\NormalTok{habitat)}

\CommentTok{\# build taxa x samples incidence}
\NormalTok{X }\OtherTok{\textless{}{-}} \FunctionTok{as}\NormalTok{(}\FunctionTok{otu\_table}\NormalTok{(ps\_pd), }\StringTok{"matrix"}\NormalTok{)}
\ControlFlowTok{if}\NormalTok{ (}\SpecialCharTok{!}\FunctionTok{taxa\_are\_rows}\NormalTok{(ps\_pd)) X }\OtherTok{\textless{}{-}} \FunctionTok{t}\NormalTok{(X)  }\CommentTok{\# taxa x samples}

\NormalTok{sd2 }\OtherTok{\textless{}{-}} \FunctionTok{as.data.frame}\NormalTok{(}\FunctionTok{sample\_data}\NormalTok{(ps\_pd))}

\NormalTok{inc\_by\_hab\_pd }\OtherTok{\textless{}{-}} \FunctionTok{lapply}\NormalTok{(}\FunctionTok{split}\NormalTok{(}\FunctionTok{rownames}\NormalTok{(sd2), sd2}\SpecialCharTok{$}\NormalTok{habitat), }\ControlFlowTok{function}\NormalTok{(samps)\{}
\NormalTok{  M }\OtherTok{\textless{}{-}}\NormalTok{ X[, }\FunctionTok{colnames}\NormalTok{(X) }\SpecialCharTok{\%in\%}\NormalTok{ samps, drop }\OtherTok{=} \ConstantTok{FALSE}\NormalTok{]}
\NormalTok{  M[M }\SpecialCharTok{\textgreater{}} \DecValTok{0}\NormalTok{] }\OtherTok{\textless{}{-}} \DecValTok{1}
  \FunctionTok{storage.mode}\NormalTok{(M) }\OtherTok{\textless{}{-}} \StringTok{"numeric"}
\NormalTok{  M}
\NormalTok{\})}

\NormalTok{inc\_by\_hab\_pd }\OtherTok{\textless{}{-}}\NormalTok{ inc\_by\_hab\_pd[}\FunctionTok{names}\NormalTok{(inc\_by\_hab\_pd) }\SpecialCharTok{\%in\%} \FunctionTok{names}\NormalTok{(labs\_map)]}
\NormalTok{inc\_by\_hab\_pd }\OtherTok{\textless{}{-}}\NormalTok{ inc\_by\_hab\_pd[}\FunctionTok{vapply}\NormalTok{(inc\_by\_hab\_pd, }\ControlFlowTok{function}\NormalTok{(M) }\FunctionTok{nrow}\NormalTok{(M) }\SpecialCharTok{\textgreater{}} \DecValTok{0} \SpecialCharTok{\&\&} \FunctionTok{ncol}\NormalTok{(M) }\SpecialCharTok{\textgreater{}} \DecValTok{0}\NormalTok{, }\ConstantTok{TRUE}\NormalTok{)]}

\CommentTok{\# Tree: prune to observed taxa in incidence matrices}
\NormalTok{tr }\OtherTok{\textless{}{-}} \FunctionTok{phy\_tree}\NormalTok{(ps\_pd)}
\NormalTok{keep\_tips }\OtherTok{\textless{}{-}} \FunctionTok{intersect}\NormalTok{(tr}\SpecialCharTok{$}\NormalTok{tip.label, }\FunctionTok{unique}\NormalTok{(}\FunctionTok{unlist}\NormalTok{(}\FunctionTok{lapply}\NormalTok{(inc\_by\_hab\_pd, rownames))))}
\FunctionTok{stopifnot}\NormalTok{(}\FunctionTok{length}\NormalTok{(keep\_tips) }\SpecialCharTok{\textgreater{}=} \DecValTok{2}\NormalTok{)}

\NormalTok{tr2 }\OtherTok{\textless{}{-}}\NormalTok{ ape}\SpecialCharTok{::}\FunctionTok{keep.tip}\NormalTok{(tr, keep\_tips)}
\NormalTok{tr2}\SpecialCharTok{$}\NormalTok{node.label }\OtherTok{\textless{}{-}} \ConstantTok{NULL}
\ControlFlowTok{if}\NormalTok{ (}\SpecialCharTok{!}\NormalTok{ape}\SpecialCharTok{::}\FunctionTok{is.rooted}\NormalTok{(tr2)) tr2 }\OtherTok{\textless{}{-}}\NormalTok{ phangorn}\SpecialCharTok{::}\FunctionTok{midpoint}\NormalTok{(tr2)}

\CommentTok{\# Align matrices to final tip set (same taxa order)}
\NormalTok{tipset }\OtherTok{\textless{}{-}}\NormalTok{ tr2}\SpecialCharTok{$}\NormalTok{tip.label}
\NormalTok{inc\_by\_hab\_pd }\OtherTok{\textless{}{-}} \FunctionTok{lapply}\NormalTok{(inc\_by\_hab\_pd, }\ControlFlowTok{function}\NormalTok{(M) M[}\FunctionTok{rownames}\NormalTok{(M) }\SpecialCharTok{\%in\%}\NormalTok{ tipset, , }\AttributeTok{drop =} \ConstantTok{FALSE}\NormalTok{])}
\NormalTok{inc\_by\_hab\_pd }\OtherTok{\textless{}{-}}\NormalTok{ inc\_by\_hab\_pd[}\FunctionTok{vapply}\NormalTok{(inc\_by\_hab\_pd, }\ControlFlowTok{function}\NormalTok{(M) }\FunctionTok{nrow}\NormalTok{(M) }\SpecialCharTok{\textgreater{}} \DecValTok{0} \SpecialCharTok{\&\&} \FunctionTok{ncol}\NormalTok{(M) }\SpecialCharTok{\textgreater{}} \DecValTok{0}\NormalTok{, }\ConstantTok{TRUE}\NormalTok{)]}

\NormalTok{out\_PD }\OtherTok{\textless{}{-}} \FunctionTok{iNEXT3D}\NormalTok{(}
  \AttributeTok{data      =}\NormalTok{ inc\_by\_hab\_pd,}
  \AttributeTok{diversity =} \StringTok{"PD"}\NormalTok{,}
  \AttributeTok{q         =} \FunctionTok{c}\NormalTok{(}\DecValTok{0}\NormalTok{,}\DecValTok{1}\NormalTok{,}\DecValTok{2}\NormalTok{),}
  \AttributeTok{datatype  =} \StringTok{"incidence\_raw"}\NormalTok{,}
  \AttributeTok{nboot     =} \DecValTok{500}\NormalTok{,}
  \AttributeTok{PDtree    =}\NormalTok{ tr2,}
  \AttributeTok{PDtype    =} \StringTok{"meanPD"}
\NormalTok{)}

\CommentTok{\# Plot}
\NormalTok{p\_PD }\OtherTok{\textless{}{-}} \FunctionTok{ggiNEXT3D}\NormalTok{(out\_PD, }\AttributeTok{type =} \DecValTok{1}\NormalTok{, }\AttributeTok{facet.var =} \StringTok{"Order.q"}\NormalTok{) }\SpecialCharTok{+}
  \FunctionTok{labs}\NormalTok{(}\AttributeTok{x =} \StringTok{"Sampling units"}\NormalTok{, }\AttributeTok{y =} \StringTok{"Phylogenetic diversity (meanPD)"}\NormalTok{) }\SpecialCharTok{+}
  \FunctionTok{theme\_minimal}\NormalTok{(}\AttributeTok{base\_size =} \DecValTok{14}\NormalTok{) }\SpecialCharTok{+}
  \FunctionTok{theme}\NormalTok{(}\AttributeTok{legend.title =} \FunctionTok{element\_blank}\NormalTok{())}

\FunctionTok{ggsave}\NormalTok{(}
  \AttributeTok{filename =} \FunctionTok{file.path}\NormalTok{(out\_fig\_dir, }\StringTok{"Fig\_4A\_iNEXT3D\_PD.png"}\NormalTok{),}
  \AttributeTok{plot =}\NormalTok{ p\_PD, }\AttributeTok{width =} \DecValTok{10}\NormalTok{, }\AttributeTok{height =} \DecValTok{6}\NormalTok{, }\AttributeTok{dpi =} \DecValTok{800}\NormalTok{, }\AttributeTok{bg =} \StringTok{"white"}
\NormalTok{)}

\CommentTok{\# Table S11: PD summary lives in $PDAsyEst (same logic as TD)}
\NormalTok{tab\_s11 }\OtherTok{\textless{}{-}} \FunctionTok{as.data.frame}\NormalTok{(out\_PD}\SpecialCharTok{$}\NormalTok{PDAsyEst) }\SpecialCharTok{\%\textgreater{}\%}
  \FunctionTok{rename}\NormalTok{(}\AttributeTok{Habitat =}\NormalTok{ Assemblage)}

\CommentTok{\# Relabel habitats to capitalized versions (same mapping)}
\NormalTok{tab\_s11}\SpecialCharTok{$}\NormalTok{Habitat }\OtherTok{\textless{}{-}} \FunctionTok{as.character}\NormalTok{(tab\_s11}\SpecialCharTok{$}\NormalTok{Habitat)}
\NormalTok{tab\_s11}\SpecialCharTok{$}\NormalTok{Habitat }\OtherTok{\textless{}{-}}\NormalTok{ labs\_map[tab\_s11}\SpecialCharTok{$}\NormalTok{Habitat]}
\NormalTok{tab\_s11}\SpecialCharTok{$}\NormalTok{Habitat }\OtherTok{\textless{}{-}} \FunctionTok{factor}\NormalTok{(tab\_s11}\SpecialCharTok{$}\NormalTok{Habitat, }\AttributeTok{levels =}\NormalTok{ lvl\_order)}

\FunctionTok{write.csv}\NormalTok{(}
\NormalTok{  tab\_s11,}
  \FunctionTok{file.path}\NormalTok{(out\_tab\_dir, }\StringTok{"Table\_S12\_iNEXT3D\_phylogenetic\_diversity.csv"}\NormalTok{),}
  \AttributeTok{row.names =} \ConstantTok{FALSE}
\NormalTok{)}
\end{Highlighting}
\end{Shaded}

\subsection{Nearest Taxon Index (NTI) and Net Relatedness Index
(NRI)}\label{nearest-taxon-index-nti-and-net-relatedness-index-nri}

Phylogenetic community structure was quantified using Net Relatedness
Index (NRI; based on MPD) and Nearest Taxon Index (NTI; based on MNTD),
calculated under a taxa-label null model with 999 randomizations using
both abundance-weighted (aw) and presence--absence (pa) versions.
Habitat-level summaries are reported as median (IQR) (Table S13).
Habitat and site effects on abundance-weighted metrics were evaluated
with linear models (habitat + site; Table S14). Model diagnostics for
the NTI\_aw model are available in Fig. S10.

\textbf{Table S13}: NTI and NRI across habitats.

\begin{longtable}[]{@{}lrllll@{}}
\caption{Table S13. Median (IQR) of NRI and NTI across habitats,
calculated using abundance-weighted (aw) and presence--absence (pa)
metrics. Positive values indicate phylogenetic clustering relative to
null expectations.}\tabularnewline
\toprule\noalign{}
Habitat & n & NRI\_aw & NTI\_aw & NRI\_pa & NTI\_pa \\
\midrule\noalign{}
\endfirsthead
\toprule\noalign{}
Habitat & n & NRI\_aw & NTI\_aw & NRI\_pa & NTI\_pa \\
\midrule\noalign{}
\endhead
\bottomrule\noalign{}
\endlastfoot
forest & 25 & 0.991 (3.02) & 3.053 (1.70) & 0.711 (2.19) & 5.764
(2.13) \\
paramo & 15 & 1.435 (1.85) & 3.277 (1.62) & 0.831 (2.81) & 4.930
(2.06) \\
subparamo & 20 & 1.675 (1.77) & 2.831 (1.40) & 0.395 (1.50) & 4.876
(2.49) \\
\end{longtable}

\textbf{Table S14}: ANOVA summaries for NTI and NRI

\begin{longtable}[]{@{}llrrr@{}}
\caption{Table S14. ANOVA summaries for linear models of
abundance-weighted NTI and NRI (habitat + site).}\tabularnewline
\toprule\noalign{}
Response & Predictor & df & F & p \\
\midrule\noalign{}
\endfirsthead
\toprule\noalign{}
Response & Predictor & df & F & p \\
\midrule\noalign{}
\endhead
\bottomrule\noalign{}
\endlastfoot
NTI\_aw & habitat & 2 & 0.5106 & 0.6030 \\
NTI\_aw & site & 3 & 1.8164 & 0.1551 \\
NTI\_aw & Residuals & 54 & NA & NA \\
NRI\_aw & habitat & 2 & 1.2729 & 0.2883 \\
NRI\_aw & site & 3 & 3.4206 & 0.0235 \\
NRI\_aw & Residuals & 54 & NA & NA \\
\end{longtable}

\includegraphics{figures/Fig_S11_NTI_aw_diagnostics.png}

\textbf{Figure S10}: Diagnostic plots for the NTI\_aw linear model.

\begin{Shaded}
\begin{Highlighting}[]
\CommentTok{\# Key outputs saved:}
\CommentTok{\#   {-} figures/Fig\_S10\_NTI\_aw\_diagnostics.png}
\CommentTok{\#   {-} tables/Table\_S13\_NRI\_NTI\_median\_IQR.csv}
\CommentTok{\#   {-} tables/Table\_S14\_NRI\_NTI\_ANOVA.csv}

\FunctionTok{library}\NormalTok{(phyloseq)}
\FunctionTok{library}\NormalTok{(picante)}
\FunctionTok{library}\NormalTok{(ape)}
\FunctionTok{library}\NormalTok{(phangorn)}
\FunctionTok{library}\NormalTok{(dplyr)}
\FunctionTok{library}\NormalTok{(ggplot2)}
\FunctionTok{library}\NormalTok{(patchwork)}

\FunctionTok{set.seed}\NormalTok{(}\DecValTok{1}\NormalTok{)}

\NormalTok{ps }\OtherTok{\textless{}{-}}\NormalTok{ ps\_tree }\CommentTok{\#phyloseq object with tree}

\FunctionTok{stopifnot}\NormalTok{(}\FunctionTok{exists}\NormalTok{(}\StringTok{"ps"}\NormalTok{))}
\NormalTok{tr2 }\OtherTok{\textless{}{-}} \FunctionTok{phy\_tree}\NormalTok{(ps)}
\FunctionTok{stopifnot}\NormalTok{(}\SpecialCharTok{!}\FunctionTok{is.null}\NormalTok{(tr2))}

\CommentTok{\# Root tree if needed}
\ControlFlowTok{if}\NormalTok{ (}\SpecialCharTok{!}\NormalTok{ape}\SpecialCharTok{::}\FunctionTok{is.rooted}\NormalTok{(tr2)) tr2 }\OtherTok{\textless{}{-}}\NormalTok{ phangorn}\SpecialCharTok{::}\FunctionTok{midpoint}\NormalTok{(tr2)}

\DocumentationTok{\#\# 1) samples x taxa abundance matrix}
\NormalTok{comm }\OtherTok{\textless{}{-}} \FunctionTok{as}\NormalTok{(}\FunctionTok{otu\_table}\NormalTok{(ps), }\StringTok{"matrix"}\NormalTok{)}
\ControlFlowTok{if}\NormalTok{ (}\FunctionTok{taxa\_are\_rows}\NormalTok{(ps)) comm }\OtherTok{\textless{}{-}} \FunctionTok{t}\NormalTok{(comm)}

\DocumentationTok{\#\# 2) align taxa with tree tips}
\NormalTok{keep\_taxa }\OtherTok{\textless{}{-}} \FunctionTok{intersect}\NormalTok{(}\FunctionTok{colnames}\NormalTok{(comm), tr2}\SpecialCharTok{$}\NormalTok{tip.label)}
\NormalTok{comm }\OtherTok{\textless{}{-}}\NormalTok{ comm[, keep\_taxa, drop }\OtherTok{=} \ConstantTok{FALSE}\NormalTok{]}
\NormalTok{tr2  }\OtherTok{\textless{}{-}}\NormalTok{ ape}\SpecialCharTok{::}\FunctionTok{keep.tip}\NormalTok{(tr2, keep\_taxa)}

\DocumentationTok{\#\# 3) optional: drop ultra{-}rare taxa }
\NormalTok{min\_total\_taxon\_reads }\OtherTok{\textless{}{-}} \DecValTok{10}
\NormalTok{keep\_taxa2 }\OtherTok{\textless{}{-}} \FunctionTok{colSums}\NormalTok{(comm) }\SpecialCharTok{\textgreater{}=}\NormalTok{ min\_total\_taxon\_reads}
\NormalTok{comm }\OtherTok{\textless{}{-}}\NormalTok{ comm[, keep\_taxa2, drop }\OtherTok{=} \ConstantTok{FALSE}\NormalTok{]}
\NormalTok{tr2  }\OtherTok{\textless{}{-}}\NormalTok{ ape}\SpecialCharTok{::}\FunctionTok{keep.tip}\NormalTok{(tr2, }\FunctionTok{colnames}\NormalTok{(comm))}

\DocumentationTok{\#\# 4) metadata aligned to comm}
\NormalTok{md }\OtherTok{\textless{}{-}} \FunctionTok{as}\NormalTok{(}\FunctionTok{sample\_data}\NormalTok{(ps), }\StringTok{"data.frame"}\NormalTok{)}
\NormalTok{md }\OtherTok{\textless{}{-}}\NormalTok{ md[}\FunctionTok{rownames}\NormalTok{(comm), , drop }\OtherTok{=} \ConstantTok{FALSE}\NormalTok{]}
\NormalTok{md}\SpecialCharTok{$}\NormalTok{habitat }\OtherTok{\textless{}{-}} \FunctionTok{droplevels}\NormalTok{(}\FunctionTok{factor}\NormalTok{(md}\SpecialCharTok{$}\NormalTok{habitat))}
\NormalTok{md}\SpecialCharTok{$}\NormalTok{site    }\OtherTok{\textless{}{-}} \FunctionTok{droplevels}\NormalTok{(}\FunctionTok{factor}\NormalTok{(md}\SpecialCharTok{$}\NormalTok{site))}

\DocumentationTok{\#\# 5) phylogenetic distances among taxa}
\NormalTok{dist\_phy }\OtherTok{\textless{}{-}} \FunctionTok{cophenetic}\NormalTok{(tr2)}

\DocumentationTok{\#\# 6) null model tests}
\NormalTok{mpd\_aw  }\OtherTok{\textless{}{-}} \FunctionTok{ses.mpd}\NormalTok{ (comm, dist\_phy, }\AttributeTok{null.model =} \StringTok{"taxa.labels"}\NormalTok{,}
                    \AttributeTok{runs =} \DecValTok{999}\NormalTok{, }\AttributeTok{abundance.weighted =} \ConstantTok{TRUE}\NormalTok{)}
\NormalTok{mntd\_aw }\OtherTok{\textless{}{-}} \FunctionTok{ses.mntd}\NormalTok{(comm, dist\_phy, }\AttributeTok{null.model =} \StringTok{"taxa.labels"}\NormalTok{,}
                    \AttributeTok{runs =} \DecValTok{999}\NormalTok{, }\AttributeTok{abundance.weighted =} \ConstantTok{TRUE}\NormalTok{)}

\NormalTok{comm\_pa }\OtherTok{\textless{}{-}} \DecValTok{1} \SpecialCharTok{*}\NormalTok{ (comm }\SpecialCharTok{\textgreater{}} \DecValTok{0}\NormalTok{)}
\NormalTok{mpd\_pa  }\OtherTok{\textless{}{-}} \FunctionTok{ses.mpd}\NormalTok{ (comm\_pa, dist\_phy, }\AttributeTok{null.model =} \StringTok{"taxa.labels"}\NormalTok{,}
                    \AttributeTok{runs =} \DecValTok{999}\NormalTok{, }\AttributeTok{abundance.weighted =} \ConstantTok{FALSE}\NormalTok{)}
\NormalTok{mntd\_pa }\OtherTok{\textless{}{-}} \FunctionTok{ses.mntd}\NormalTok{(comm\_pa, dist\_phy, }\AttributeTok{null.model =} \StringTok{"taxa.labels"}\NormalTok{,}
                    \AttributeTok{runs =} \DecValTok{999}\NormalTok{, }\AttributeTok{abundance.weighted =} \ConstantTok{FALSE}\NormalTok{)}

\DocumentationTok{\#\# 7) per{-}sample output table}
\NormalTok{out }\OtherTok{\textless{}{-}} \FunctionTok{data.frame}\NormalTok{(}
  \AttributeTok{sample =} \FunctionTok{rownames}\NormalTok{(comm),}
  \AttributeTok{NRI\_aw =} \SpecialCharTok{{-}}\NormalTok{mpd\_aw}\SpecialCharTok{$}\NormalTok{mpd.obs.z,}
  \AttributeTok{NTI\_aw =} \SpecialCharTok{{-}}\NormalTok{mntd\_aw}\SpecialCharTok{$}\NormalTok{mntd.obs.z,}
  \AttributeTok{NRI\_pa =} \SpecialCharTok{{-}}\NormalTok{mpd\_pa}\SpecialCharTok{$}\NormalTok{mpd.obs.z,}
  \AttributeTok{NTI\_pa =} \SpecialCharTok{{-}}\NormalTok{mntd\_pa}\SpecialCharTok{$}\NormalTok{mntd.obs.z,}
  \AttributeTok{stringsAsFactors =} \ConstantTok{FALSE}
\NormalTok{)}
\FunctionTok{rownames}\NormalTok{(out) }\OtherTok{\textless{}{-}}\NormalTok{ out}\SpecialCharTok{$}\NormalTok{sample}

\CommentTok{\# Drop any sample with NA (e.g., too few taxa after filtering)}
\NormalTok{out }\OtherTok{\textless{}{-}}\NormalTok{ out[}\FunctionTok{complete.cases}\NormalTok{(out[, }\FunctionTok{c}\NormalTok{(}\StringTok{"NRI\_aw"}\NormalTok{,}\StringTok{"NTI\_aw"}\NormalTok{,}\StringTok{"NRI\_pa"}\NormalTok{,}\StringTok{"NTI\_pa"}\NormalTok{)]), ]}
\NormalTok{out }\OtherTok{\textless{}{-}} \FunctionTok{cbind}\NormalTok{(out, md[}\FunctionTok{rownames}\NormalTok{(out), }\FunctionTok{c}\NormalTok{(}\StringTok{"habitat"}\NormalTok{,}\StringTok{"site"}\NormalTok{), }\AttributeTok{drop =} \ConstantTok{FALSE}\NormalTok{])}
\NormalTok{out}\SpecialCharTok{$}\NormalTok{habitat }\OtherTok{\textless{}{-}} \FunctionTok{droplevels}\NormalTok{(}\FunctionTok{factor}\NormalTok{(out}\SpecialCharTok{$}\NormalTok{habitat))}
\NormalTok{out}\SpecialCharTok{$}\NormalTok{site    }\OtherTok{\textless{}{-}} \FunctionTok{droplevels}\NormalTok{(}\FunctionTok{factor}\NormalTok{(out}\SpecialCharTok{$}\NormalTok{site))}

\DocumentationTok{\#\# 8) Table S12: median (IQR) per habitat, formatted as "median (IQR)"}
\NormalTok{fmt\_med\_iqr }\OtherTok{\textless{}{-}} \ControlFlowTok{function}\NormalTok{(x) }\FunctionTok{sprintf}\NormalTok{(}\StringTok{"\%.3f (\%.2f)"}\NormalTok{, }\FunctionTok{median}\NormalTok{(x, }\AttributeTok{na.rm=}\ConstantTok{TRUE}\NormalTok{), }\FunctionTok{IQR}\NormalTok{(x, }\AttributeTok{na.rm=}\ConstantTok{TRUE}\NormalTok{))}

\NormalTok{tab\_s12 }\OtherTok{\textless{}{-}}\NormalTok{ out }\SpecialCharTok{\%\textgreater{}\%}
  \FunctionTok{group\_by}\NormalTok{(habitat) }\SpecialCharTok{\%\textgreater{}\%}
  \FunctionTok{summarise}\NormalTok{(}
    \AttributeTok{n =} \FunctionTok{n}\NormalTok{(),}
    \AttributeTok{NRI\_aw =} \FunctionTok{fmt\_med\_iqr}\NormalTok{(NRI\_aw),}
    \AttributeTok{NTI\_aw =} \FunctionTok{fmt\_med\_iqr}\NormalTok{(NTI\_aw),}
    \AttributeTok{NRI\_pa =} \FunctionTok{fmt\_med\_iqr}\NormalTok{(NRI\_pa),}
    \AttributeTok{NTI\_pa =} \FunctionTok{fmt\_med\_iqr}\NormalTok{(NTI\_pa),}
    \AttributeTok{.groups =} \StringTok{"drop"}
\NormalTok{  ) }\SpecialCharTok{\%\textgreater{}\%}
  \FunctionTok{rename}\NormalTok{(}\AttributeTok{Habitat =}\NormalTok{ habitat)}

\FunctionTok{write.csv}\NormalTok{(tab\_s12, }\FunctionTok{file.path}\NormalTok{(out\_tab\_dir, }\StringTok{"Table\_S13\_NRI\_NTI\_median\_IQR.csv"}\NormalTok{), }\AttributeTok{row.names =} \ConstantTok{FALSE}\NormalTok{)}

\DocumentationTok{\#\# 9) Table S13: ANOVA summaries (NTI\_aw and NRI\_aw models)}
\NormalTok{m\_NTI\_aw }\OtherTok{\textless{}{-}} \FunctionTok{lm}\NormalTok{(NTI\_aw }\SpecialCharTok{\textasciitilde{}}\NormalTok{ habitat }\SpecialCharTok{+}\NormalTok{ site, }\AttributeTok{data =}\NormalTok{ out)}
\NormalTok{m\_NRI\_aw }\OtherTok{\textless{}{-}} \FunctionTok{lm}\NormalTok{(NRI\_aw }\SpecialCharTok{\textasciitilde{}}\NormalTok{ habitat }\SpecialCharTok{+}\NormalTok{ site, }\AttributeTok{data =}\NormalTok{ out)}

\NormalTok{a\_NTI }\OtherTok{\textless{}{-}} \FunctionTok{as.data.frame}\NormalTok{(}\FunctionTok{anova}\NormalTok{(m\_NTI\_aw))}
\NormalTok{a\_NRI }\OtherTok{\textless{}{-}} \FunctionTok{as.data.frame}\NormalTok{(}\FunctionTok{anova}\NormalTok{(m\_NRI\_aw))}

\NormalTok{tab\_s13 }\OtherTok{\textless{}{-}} \FunctionTok{bind\_rows}\NormalTok{(}
  \FunctionTok{data.frame}\NormalTok{(}\AttributeTok{Response =} \StringTok{"NTI\_aw"}\NormalTok{, }\AttributeTok{Predictor =} \FunctionTok{rownames}\NormalTok{(a\_NTI), }\AttributeTok{df =}\NormalTok{ a\_NTI}\SpecialCharTok{$}\NormalTok{Df, }\AttributeTok{F =}\NormalTok{ a\_NTI}\SpecialCharTok{$}\StringTok{\textasciigrave{}}\AttributeTok{F value}\StringTok{\textasciigrave{}}\NormalTok{, }\AttributeTok{p =}\NormalTok{ a\_NTI}\SpecialCharTok{$}\StringTok{\textasciigrave{}}\AttributeTok{Pr(\textgreater{}F)}\StringTok{\textasciigrave{}}\NormalTok{),}
  \FunctionTok{data.frame}\NormalTok{(}\AttributeTok{Response =} \StringTok{"NRI\_aw"}\NormalTok{, }\AttributeTok{Predictor =} \FunctionTok{rownames}\NormalTok{(a\_NRI), }\AttributeTok{df =}\NormalTok{ a\_NRI}\SpecialCharTok{$}\NormalTok{Df, }\AttributeTok{F =}\NormalTok{ a\_NRI}\SpecialCharTok{$}\StringTok{\textasciigrave{}}\AttributeTok{F value}\StringTok{\textasciigrave{}}\NormalTok{, }\AttributeTok{p =}\NormalTok{ a\_NRI}\SpecialCharTok{$}\StringTok{\textasciigrave{}}\AttributeTok{Pr(\textgreater{}F)}\StringTok{\textasciigrave{}}\NormalTok{)}
\NormalTok{) }\SpecialCharTok{\%\textgreater{}\%}
  \FunctionTok{filter}\NormalTok{(Predictor }\SpecialCharTok{\%in\%} \FunctionTok{c}\NormalTok{(}\StringTok{"habitat"}\NormalTok{,}\StringTok{"site"}\NormalTok{,}\StringTok{"Residuals"}\NormalTok{))}

\FunctionTok{write.csv}\NormalTok{(tab\_s13, }\FunctionTok{file.path}\NormalTok{(out\_tab\_dir, }\StringTok{"Table\_S14\_NRI\_NTI\_ANOVA.csv"}\NormalTok{), }\AttributeTok{row.names =} \ConstantTok{FALSE}\NormalTok{)}

\DocumentationTok{\#\# 10) Figure S11: diagnostics for NTI\_aw model}
\NormalTok{diagdf }\OtherTok{\textless{}{-}} \FunctionTok{data.frame}\NormalTok{(}
  \AttributeTok{fitted =} \FunctionTok{fitted}\NormalTok{(m\_NTI\_aw),}
  \AttributeTok{resid  =} \FunctionTok{resid}\NormalTok{(m\_NTI\_aw),}
  \AttributeTok{stdres =} \FunctionTok{rstandard}\NormalTok{(m\_NTI\_aw),}
  \AttributeTok{cooks  =} \FunctionTok{cooks.distance}\NormalTok{(m\_NTI\_aw),}
  \AttributeTok{hat    =} \FunctionTok{hatvalues}\NormalTok{(m\_NTI\_aw)}
\NormalTok{)}

\NormalTok{p1 }\OtherTok{\textless{}{-}} \FunctionTok{ggplot}\NormalTok{(diagdf, }\FunctionTok{aes}\NormalTok{(fitted, resid)) }\SpecialCharTok{+}
  \FunctionTok{geom\_hline}\NormalTok{(}\AttributeTok{yintercept =} \DecValTok{0}\NormalTok{, }\AttributeTok{linetype =} \StringTok{"dashed"}\NormalTok{) }\SpecialCharTok{+}
  \FunctionTok{geom\_point}\NormalTok{(}\AttributeTok{alpha =} \FloatTok{0.7}\NormalTok{, }\AttributeTok{size =} \FloatTok{1.8}\NormalTok{) }\SpecialCharTok{+}
  \FunctionTok{geom\_smooth}\NormalTok{(}\AttributeTok{method =} \StringTok{"loess"}\NormalTok{, }\AttributeTok{se =} \ConstantTok{FALSE}\NormalTok{, }\AttributeTok{linewidth =} \FloatTok{0.5}\NormalTok{) }\SpecialCharTok{+}
  \FunctionTok{labs}\NormalTok{(}\AttributeTok{x =} \StringTok{"Fitted values"}\NormalTok{, }\AttributeTok{y =} \StringTok{"Residuals"}\NormalTok{, }\AttributeTok{title =} \StringTok{"Residuals vs fitted"}\NormalTok{) }\SpecialCharTok{+}
  \FunctionTok{theme\_bw}\NormalTok{(}\AttributeTok{base\_size =} \DecValTok{10}\NormalTok{)}

\NormalTok{p2 }\OtherTok{\textless{}{-}} \FunctionTok{ggplot}\NormalTok{(diagdf, }\FunctionTok{aes}\NormalTok{(}\AttributeTok{sample =}\NormalTok{ stdres)) }\SpecialCharTok{+}
  \FunctionTok{stat\_qq}\NormalTok{(}\AttributeTok{alpha =} \FloatTok{0.7}\NormalTok{, }\AttributeTok{size =} \FloatTok{1.5}\NormalTok{) }\SpecialCharTok{+}
  \FunctionTok{stat\_qq\_line}\NormalTok{() }\SpecialCharTok{+}
  \FunctionTok{labs}\NormalTok{(}\AttributeTok{title =} \StringTok{"Normal Q–Q (standardized residuals)"}\NormalTok{) }\SpecialCharTok{+}
  \FunctionTok{theme\_bw}\NormalTok{(}\AttributeTok{base\_size =} \DecValTok{10}\NormalTok{)}

\NormalTok{p3 }\OtherTok{\textless{}{-}} \FunctionTok{ggplot}\NormalTok{(diagdf, }\FunctionTok{aes}\NormalTok{(}\FunctionTok{seq\_along}\NormalTok{(cooks), cooks)) }\SpecialCharTok{+}
  \FunctionTok{geom\_point}\NormalTok{(}\AttributeTok{alpha =} \FloatTok{0.7}\NormalTok{, }\AttributeTok{size =} \FloatTok{1.8}\NormalTok{) }\SpecialCharTok{+}
  \FunctionTok{geom\_hline}\NormalTok{(}\AttributeTok{yintercept =} \FloatTok{0.5}\NormalTok{, }\AttributeTok{linetype =} \StringTok{"dashed"}\NormalTok{) }\SpecialCharTok{+}
  \FunctionTok{labs}\NormalTok{(}\AttributeTok{x =} \StringTok{"Observation index"}\NormalTok{, }\AttributeTok{y =} \StringTok{"Cook\textquotesingle{}s distance"}\NormalTok{, }\AttributeTok{title =} \StringTok{"Influence (Cook\textquotesingle{}s distance)"}\NormalTok{) }\SpecialCharTok{+}
  \FunctionTok{theme\_bw}\NormalTok{(}\AttributeTok{base\_size =} \DecValTok{10}\NormalTok{)}

\NormalTok{p4 }\OtherTok{\textless{}{-}} \FunctionTok{ggplot}\NormalTok{(diagdf, }\FunctionTok{aes}\NormalTok{(}\FunctionTok{seq\_along}\NormalTok{(hat), hat)) }\SpecialCharTok{+}
  \FunctionTok{geom\_point}\NormalTok{(}\AttributeTok{alpha =} \FloatTok{0.7}\NormalTok{, }\AttributeTok{size =} \FloatTok{1.8}\NormalTok{) }\SpecialCharTok{+}
  \FunctionTok{labs}\NormalTok{(}\AttributeTok{x =} \StringTok{"Observation index"}\NormalTok{, }\AttributeTok{y =} \StringTok{"Leverage (hat)"}\NormalTok{, }\AttributeTok{title =} \StringTok{"Leverage"}\NormalTok{) }\SpecialCharTok{+}
  \FunctionTok{theme\_bw}\NormalTok{(}\AttributeTok{base\_size =} \DecValTok{10}\NormalTok{)}

\NormalTok{panel }\OtherTok{\textless{}{-}}\NormalTok{ (p1 }\SpecialCharTok{|}\NormalTok{ p2) }\SpecialCharTok{/}\NormalTok{ (p3 }\SpecialCharTok{|}\NormalTok{ p4) }\SpecialCharTok{+} \FunctionTok{plot\_annotation}\NormalTok{(}\AttributeTok{title =} \StringTok{"NTI\_aw model diagnostics"}\NormalTok{)}

\FunctionTok{ggsave}\NormalTok{(}
  \AttributeTok{filename =} \FunctionTok{file.path}\NormalTok{(out\_fig\_dir, }\StringTok{"Fig\_S10\_NTI\_aw\_diagnostics.png"}\NormalTok{),}
  \AttributeTok{plot =}\NormalTok{ panel,}
  \AttributeTok{width =} \DecValTok{10}\NormalTok{, }\AttributeTok{height =} \DecValTok{7}\NormalTok{, }\AttributeTok{dpi =} \DecValTok{600}\NormalTok{, }\AttributeTok{bg =} \StringTok{"white"}
\NormalTok{)}
\end{Highlighting}
\end{Shaded}

\subsection{Soil Physicochemical
parameters}\label{soil-physicochemical-parameters}

To provide environmental context for the elevational gradients sampled
in this study, we quantified bulk soil physicochemical properties from
one composite topsoil sample (0--15 cm) collected at each sampling
location (n = 12; one per location). Because soil chemistry was measured
once per location, these data are interpreted descriptively and were not
used for formal statistical inference among habitats or sites.

The full soil dataset, including all measured variables, is provided as
Table S15 and is archived in the repository as ``Data\_S3\_soil.csv''.

\hfill\break
\hfill\break
\textbf{Figure S11}: Correlation heatmap of bulk soil physicochemical
variables measured at each sampling location (n = 12). Colors indicate
Pearson correlation among soil variables and elevation.



\end{document}
